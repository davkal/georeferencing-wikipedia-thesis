%************************************************
\chapter{Foundation}\label{ch:foundation}
%************************************************

\todo{weave together important concepts for this thesis and split prior research in areas:}
\begin{todos}
    \item \fullcite{nielsen2011wikipedia}
    \item why wikipedia?
    \item wikipedia production
    \item contribution/attribution
    \item georeferencing
    \item visualization
\end{todos}

\section{Wikipedia}

\todo{this section should cover the basics to understand components of wikipedia}

Die Online-Enzyklopädie Wikipedia gibt es in über 260 Sprachvarianten, von denen die englische mit derzeit 3,6 Millionen Artikeln mit Abstand die größte ist.
Die Anzahl der Artikel in den anderen Sprachen sowie die Nutzung der jeweiligen Sprachvariante unterscheiden sich jedoch erheblich.\cite{wikistats}
Wenn ein Artikel zum selben Thema in Wikipedias unterschiedlicher Sprachen vorhanden ist, sind diese Varianten in der Regel über sogenannte Interwiki-Links untereinander verlinkt.

\todo{language editions,  chart}

Die Artikel dieser Lexika werden von Freiwilligen auf der ganzen Welt geschrieben, gemeinschaftlich korrigiert und aktualisiert.
Jede Änderung eines Artikels erzeugt eine neue Version, die der Versionsgeschichte des Artikels hinzugefügt wird und danach für alle Benutzer einsehbar ist.

\todo{article, draw nice graphic of UI}

Jeder Eintrag in der Versionsgeschichte besteht dabei aus der Textänderung, dem Datum der Version, dem Benutzer sowie einem optionalen Kommentar über den Grund der Änderung.
Jede Änderung kann mit Hilfe dieser Historie ausführlich begutachtet und bei Missfallen wieder revidiert werden. 
Dies kann mitunter sogenannte \emph{edit wars} hervorrufen, in denen neue Beiträge von Nutzern mit entgegengesetzten Standpunkten sofort wieder revidiert werden.\cite{suh2007us} 

\todo{revision history, changesets}

Die Mitarbeit an den Artikeln kann mit oder ohne vorherige Registrierung erfolgen.
Autoren, die sich registrieren, erlangen sowohl bestimmte Privilegien wie zum Beispiel das Recht, neue Einträge zu erstellen, als auch den Zugang zu Wikipedias sozialem Netzwerk:
Jeder Benutzer erhält nach der Registrierung eine \emph{user page} auf der er Informationen über sich veröffentlichen und über die er mit anderen Nutzern Kontakt aufnehmen kann.\cite{wikiwhyaccount}


\section{Contributions}\label{sec:contribution}

\todo{introduce collective authorship and name some important concepts. prior research in:}

\begin{todos}
    \item text-longevity
    \item attribution
\end{todos}

Eine Analyse der Autorschaft bis auf Satzebene innerhalb eines Artikels wird von Kramer in \cite{kramer2008wiki} erforscht.
Durch Auswertung der Versionsgeschichte lässt sich zu jedem Satz der Autor bestimmen, der dessen Hauptteil geschrieben hat.
Eine automatische Auswertung eines Artikels bis auf Wortebene wird von Adler in \cite{adler2008assigning} vorgestellt.
Sie basiert auf dem von Adler selbst entwickelten Reputationssystem \cite{adler2007content}, das Textstellen eine hohe Vertrauenswürdigkeit zuweist, die von einem vertrauenswürdigen Autor geschrieben oder mindestens einmal bearbeitet worden sind.\footnote{Basierend auf diesen beiden Arbeiten wurde die Software WikiTrust implementiert, welches die Vertrauenswürdigkeit als weiß-orange \emph{Heatmap} darstellt: zweifelhafte Textstellen werden orange hinterlegt und damit leicht erkennbar. Über ein API ist eine mit Vertrauenspunkten annotierte Version eines Artikels abrufbar: \url{http://www.wikitrust.net/vandalism-api}\label{wikitrust}}

Für eine Analyse der Artikel bis auf Satzebene werden Algorithmen wie in \cite{kramer2008wiki} auf ihre Anwendbarkeit untersucht.


\section{Georeferences}\label{sec:georeference}

\todo{explain this intermediate step to assign a location to a contribution}

\begin{todos}
    \item pick up where he left off: \fullcite{hardy2011volunteered} 
    \\ He's more about the how, I'm about the where. Key findings:
    \item ''I find that as a group, anonymous contributors write about fewer places than registered contributors, despite outnumbering them five-to-one.'' \cite{hardy2011volunteered}
    \item ''I find that anonymous contributors are more likely to write about
nearby places, and that the geographic effects fit an exponential distance decay
function.'' \cite{hardy2011volunteered}
    \item ''Combined approaches (i.e., where quantitative spatial analysis models
are calibrated with surveyed locations) may prove useful.''  \cite[p. 85]{hardy2011volunteered}
    \item WikiScanner
    \item Erik Zachte's: Wikipedia edits visualized\footnote{\url{http://infodisiac.com/blog/2011/05/wikipedia-edits-visualized/}}
    \item Indirect approach \fullcite{lieberman2009you}
\end{todos}

Zur Bestimmung der Herkunft eines Autors bietet Wikipedia zwei direkte Ansätze: 
Für jeden Beitrag eines nicht registrierten Benutzers wird die IP-Adresse gespeichert, über die er Zugang zum Internet erlangt hat. 
Die registrierten Nutzer können jedoch auf ihrer \emph{user page} Informationen über ihre Person entweder als Freitext oder strukturiert in \emph{user boxes} veröffentlichen.

Ein zusätzlicher, indirekter Ansatz für die Bestimmung der Herkunft eines Nutzers wird von Lieberman in \emph{You are where you edit: Locating Wikipedia users through edit histories}\cite{lieberman2009you} beschrieben.
Er basiert auf der Annahme, dass ein Nutzer mit Vorliebe an Artikeln über Orte in seiner geographischen Nähe mitarbeitet. 
Diese Artikel sind in der Regel mit geographischen Koordinaten versehen und erlauben so eine sehr grobe Bestimmung des Aufenthaltsortes und dessen Visualisierung auf einer Landkarte.


\section{Visualization}\label{sec:visualization}

\todo{Write about prior works of visualizing the aspects of attribution and georeference}

\begin{todos}
    \item Erik Zachte's: Wikipedia edits visualized\footnote{\url{http://infodisiac.com/blog/2011/05/wikipedia-edits-visualized/}}
    \item Wikitrust \ref{wikitrust}
\end{todos}
