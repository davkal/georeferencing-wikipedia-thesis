%************************************************
\chapter{Foundation}\label{ch:foundation}
%************************************************

Wikipedia is a phenomenon that has attracted researchers across all fields, notably computer science and sociology, who have written over 1,000 reports on the subject.\cite{nielsen2011wikipedia}
These roughly fall into three categories: 
\begin{description}
\item[Content production] Covering all aspects of voluntary production such as motivation, collaboration, coverage and bias, quality and vandalism, actuality and geography.
\item[Information use] Treating how the resulting corpus is being used, e.g. citing in research, use in court, trendspotting, thesaurus construction or categorization.
\item[Improvement] These are studies concerned with the improvement of both the software used by Wikipedia and the content, e.g. automatic linking, bots, improved editors as well as quality and trust indicators. 
\end{description}

This thesis falls into the first category as it examines the geography of article contributions that will become part of the Wikipedia's corpus.
After a short overview of Wikipedia from a user's perspective, I will introduce its model of collective authorship and present prior research of concerning location and geography. 

\section{Wikipedia}\label{sec:wikipedia}

Wikipedia is an online encyclopedia with editions in over 260 languages.
Counting 3.6 million articles, the English version is by far the biggest.
However, other language editions differ sharply in size and usage.\cite{wikistats}
If an article covering the same topic exists in multiple languages these are connected by interwiki links.

\todo{Image: graphic of article UI}

Anyone with a browser and internet access can edit Wikipedia's articles\footnote{Some articles can be locked because of sustained vandalism or content disputes.\cite{wpprotectionpolicy}}.
In collaboration, people all over the world contribute and improve the content. 
Each edit creates a new revision of the article and is stored in the revision history.
Naturally, each article available today started from an empty page and is the result of a succession of edits.

In the revision history each entry consists of the text change, the date of submission, the user and an optional comment explaining the change.
Each revision can not only be examined by other users and but also reverted.
To minimize the potential for \term{edit wars}\cite{suh2007us} Wikipedia urges its users to discuss controversial topics on the article's talk page.

\todo{Image: revision history}

Contributions to an article can be done anonymously or as a registered user.
A registered user gains privileges like the ability to create articles or the use of the social network features in Wikipedia.
With the initial registration a \emph{user page} is created where the user is allowed to publish a profile and interact with other registered users.\cite{wikiwhyaccount}


\section{Contributions}\label{sec:contribution}

Wikipedia's articles are continuously edited by its users.
The nature of an edit can range from simple spelling or grammar correction, over improving the content of a sentence to writing or removing of paragraphs or even whole articles. 
This collective authorship makes it difficult to determine an individual author's contributions, in other words, who wrote what.

Research in this area is mostly motivated by finding ways to assign trust to individual authors.
This is based on the assumption that these authors consistently produce high quality contributions that outlive contributions of low quality.
\textcite{kramer2008wiki} devised a method to assign trust to the authors of an article by examining the wealth of information contained in the article's revision history.
They looked at an article as being a set of phrases.
The author who first wrote a sentence gets the credit for that phrase and will gain trust if it survives future edits.\footnote{\textcite{kramer2008wiki} define a sentence as an n-gram---a sequence of n words---and use a sliding window model to follow it across revisions to prevent simple rearrangements of text from counting as a new sentence.}

This survival of text was first 
\begin{todos}
    \item text-longevity
    \item attribution
\end{todos}

Eine automatische Auswertung eines Artikels bis auf Wortebene wird von Adler in \cite{adler2008assigning} vorgestellt.
Sie basiert auf dem von Adler selbst entwickelten Reputationssystem \cite{adler2007content}, das Textstellen eine hohe Vertrauenswürdigkeit zuweist, die von einem vertrauenswürdigen Autor geschrieben oder mindestens einmal bearbeitet worden sind.\footnote{Basierend auf diesen beiden Arbeiten wurde die Software WikiTrust implementiert, welches die Vertrauenswürdigkeit als weiß-orange \emph{Heatmap} darstellt: zweifelhafte Textstellen werden orange hinterlegt und damit leicht erkennbar. Über ein API ist eine mit Vertrauenspunkten annotierte Version eines Artikels abrufbar: \url{http://www.wikitrust.net/vandalism-api}\label{wikitrust}}

Für eine Analyse der Artikel bis auf Satzebene werden Algorithmen wie in \cite{kramer2008wiki} auf ihre Anwendbarkeit untersucht.


\section{Georeferences}\label{sec:georeference}

\todo{explain this intermediate step to assign a location to a contribution}

\begin{todos}
    \item pick up where he left off: \fullcite{hardy2011volunteered} 
    \\ He's more about the how, I'm about the where. Key findings:
    \item ''I find that as a group, anonymous contributors write about fewer places than registered contributors, despite outnumbering them five-to-one.'' \cite{hardy2011volunteered}
    \item ''I find that anonymous contributors are more likely to write about
nearby places, and that the geographic effects fit an exponential distance decay
function.'' \cite{hardy2011volunteered}
    \item ''Combined approaches (i.e., where quantitative spatial analysis models
are calibrated with surveyed locations) may prove useful.''  \cite[p. 85]{hardy2011volunteered}
    \item WikiScanner
    \item Erik Zachte's: Wikipedia edits visualized\footnote{\url{http://infodisiac.com/blog/2011/05/wikipedia-edits-visualized/}}
    \item Indirect approach \fullcite{lieberman2009you}
\end{todos}

Zur Bestimmung der Herkunft eines Autors bietet Wikipedia zwei direkte Ansätze: 
Für jeden Beitrag eines nicht registrierten Benutzers wird die IP-Adresse gespeichert, über die er Zugang zum Internet erlangt hat. 
Die registrierten Nutzer können jedoch auf ihrer \emph{user page} Informationen über ihre Person entweder als Freitext oder strukturiert in \emph{user boxes} veröffentlichen.

Ein zusätzlicher, indirekter Ansatz für die Bestimmung der Herkunft eines Nutzers wird von Lieberman in \emph{You are where you edit: Locating Wikipedia users through edit histories}\cite{lieberman2009you} beschrieben.
Er basiert auf der Annahme, dass ein Nutzer mit Vorliebe an Artikeln über Orte in seiner geographischen Nähe mitarbeitet. 
Diese Artikel sind in der Regel mit geographischen Koordinaten versehen und erlauben so eine sehr grobe Bestimmung des Aufenthaltsortes und dessen Visualisierung auf einer Landkarte.


\section{Visualization}\label{sec:visualization}

\todo{Write about prior works of visualizing the aspects of attribution and georeference}

\begin{todos}
    \item Erik Zachte's: Wikipedia edits visualized\footnote{\url{http://infodisiac.com/blog/2011/05/wikipedia-edits-visualized/}}
    \item Wikitrust \ref{wikitrust}
\end{todos}
