%************************************************
\chapter{Foundation}\label{ch:foundation}
%************************************************

Wikipedia is a phenomenon that has attracted researchers across all fields, notably computer science and sociology, who have written over 1,000 reports on the subject to date.
\textcite{nielsen2011wikipedia} compiled an overview of Wikipedia research\footnote{Another resource is the Wikimedia Foundation's own directory of Wikipedia research projects at \weburl{http://meta.wikimedia.org/wiki/Research:Projects}{2011}{10}{12}.} and divides these publications into four categories: 
\begin{description}
\item[Content production] Covering all aspects of voluntary production such as motivation, collaboration, coverage and bias, quality and vandalism, actuality and geography.
\item[Information use] Treating how the resulting corpus is being used, e.g. Wikipedia citation in research, use in court, trend-spotting, natural language processing and automatic translation tools, thesaurus construction or categorization.
\item[Improvement] These are studies concerned with the improvement of both the software used by Wikipedia and the content, e.g. automatic linking, improved editors as well as quality and trust indicators. 
\item[Communication] Studies in this category look at Wikipedia as an online collaboration tool for education and research. 
\end{description}

This thesis falls into the first category, content production, as it examines the geography of article contributions that will become part of the Wikipedia's corpus.
After a short overview of Wikipedia from a user's perspective, I will introduce its model of collective authorship and present prior research of concerning location and geography. 

\section{Wikipedia}\label{sec:wikipedia}

Wikipedia is an online encyclopedia with editions in over 260 languages.
Counting 3.6 million articles, the English version is by far the biggest.
However, other language editions differ sharply in size and usage.\cite{wikistats}
If articles covering the same topic exist in other language editions, these are connected by interwiki links.

\subsection{History}

Wikipedia was officially started on 15 January 2001 by Jimmy Wales and Larry Sanger.
Wales previously founded Nupedia, a free and peer-reviewed online encyclopedia written only by experts.
However, the speed of content production was extremely low.
Wikipedia was founded as a feeder project to collectively write on articles before these entered Nupedia's review process.
Wiki\-pedia then quickly created other language editions and dwarfed its predecessor\footnote{Only 24 articles were completed in Nupedia's review process. The project was officially ended in 2003.}.\cite{wphistory}

After being mentioned on Slashdot, a technology news website, in March 2001 Wikipedia quickly attracted new users.
This tech-savvy folk created new articles at a staggering rate of 1,500 articles per month in the first year.
These articles then quickly started showing up in Google's search results, attracting even more new users.
The non-English editions grew slower but as a group accounted for 75\% of all articles in 2007.
By 2011 the combined article count passed 20 million.\cite{wphistory}

\subsection{Wikimedia Foundation}\label{sub:wmf}

Wikipedia is operated by the Wikimedia Foundation, a non-profit organization, founded in Florida on June 20, 2003.
It is completely financed by public contributions, such as donations and grants.
Individual grants can be quite substantial --- among the most generous donors are Google and the Stanton Foundation handing out respectively \$2 million and \$3.6 million in single donations.\cite{wmf}

In the Wikimedia Foundation's 2010-11 fiscal year, \$8.9 million was spent on website operations, including server hosting and software maintenance.
The rest of the \$20.0 million of total expenditures went into complementary activities such as fund raising, administration, and the support of local chapters.\cite{wmf201011}

The local chapters are self-dependent organizations set up in countries around the globe to locally promote the foundation's cause and collect donations.
The first local chapter to be created was Wikimedia Deutschland, founded in Berlin in 2004.\cite{wmf}

The individual language editions of Wikipedia are not hosted by the local chapters, however.
All of Wikipedia's content is centrally stored on servers in Tampa, Florida and in Amsterdam, Netherlands.\cite{wmf}

\subsection{Anatomy of an article}\label{sub:article}

All Wikipedia articles share a similar layout: a large content area topped by the article title.
Article titles can change over time, e.g. \emph{2011 Bahraini uprising} was renamed to \emph{2011-2012 Bahraini uprising}\footurl{http://en.wikipedia.org/wiki/2011-2012_Bahraini_uprising}{2012}{01}{23}. 
For these cases, Wikipedia has a redirecting mechanism that forwards the visitor to the final article and displays a small note below the title (\imgref{article}).

\img{article}{The article \emph{2011-2012 Bahraini uprising} viewed in a web browser on \dformat{2012}{01}{23}.}

Occasionally the content section can be topped by one or more warning boxes to inform the visitor that the article is violating an editing policy, e.g. the information of the article may be outdated because it is subject to current events.
When an article spans several sections a table of contents is added below the first introductory paragraphs.
In addition to prose, some articles feature info boxes on the right hand side.
Theses boxes show information in a structured way and can be found on articles of similar topics giving the visitor a quick glance on key information without having to read the text.

These information may include dates and geographic coordinates.
E.g. the article \term{2011-2012 Bahraini uprising} has the time interval ``14 February 2011 – ongoing'' and is tagged with the coordinates 26°01'39''N 50°33'00''E, pointing to the centre of Bahrain.
Even when no coordinates are present in the article, it still may be associated to a location.
In that case the info box just presents the place, e.g. \todo{Bahrain}, \todo{find article}, instead of providing the geographic coordinates. 

\subsection{Categories}\label{sub:categories}

At the bottom of each article is an optional list of categories that the article belongs to, e.g. the article \term{2011-2012 Bahraini uprising} belongs, among others, to ``Arab Spring by country''\footurl{http://en.wikipedia.org/wiki/Category:Arab_Spring_by_country}{2012}{01}{23} and ``2011 protests''\footurl{http://en.wikipedia.org/wiki/Category:2011_protests}{2012}{01}{23}.

Categories can not only have pages but also sub-categories, e.g. ``Arab Spring by country'' has the sub-category ``2011 Libyan civil war''\footurl{http://en.wikipedia.org/wiki/Category:2011_Libyan_civil_war}{2012}{01}{23} which in turn has 5 sub-categories and 55 pages.
The categories do not form a tree, however, for there is no restriction on the inclusion of categories --- even circular inclusions are possible.
As liberal as the topology is inclusion of articles into a category.
The category ``Arab Spring by country'' does not only contain articles covering the Arab Spring by country, but also articles about killed individuals, e.g. ``Zakariya Rashid Hassan al-Ashiri''\footurl{http://en.wikipedia.org/wiki/Zakariya_Rashid_Hassan_al-Ashiri}{2012}{01}{23}. 

\section{MediaWiki and editing}

Anyone with a browser and internet access can edit Wikipedia's articles\footnote{Some articles can be locked because of sustained vandalism or content disputes.\cite{wpprotectionpolicy}}.
In collaboration, people all over the world contribute and improve the content. 
This is made possible by MediaWiki, the the server software that makes Wikipedia a wiki. 
The software allows website visitors to add and modify the page content in the browser using \emph{wikitext}, simplified markup language.\footnote{For the syntax see \weburl{http://en.wikipedia.org/wiki/Help:Wiki_markup}{2011}{12}{12}.}
Its syntax can be used to structure a text into sections, embed images and links to other pages, much like HTML.
The syntax is explicitly kept simple to keep the entry barrier to editing low, e.g. adding an article  to a category is as easy as putting the category name at the end of the wikitext.

\subsection{Templates}\label{sub:templates}

Wikitext has a special syntax for templates\footurl{http://en.wikipedia.org/wiki/Help:Template}{2012}{01}{23}. 
These are reusable containers for text snippets and repetitive material like the info boxes described in \nameref{sub:article}.
When a template is used in a page, the server software replaces the template placeholder --- the template name surrounded by curly brackets --- with the template's content.
The content can be parameterized with key/value pairs so that, for example, an info box about countries can be used by several countries.

By using templates, the information is likely to be more structured than prose.
Each invocation of a template also renders in the same way allowing for and encouraging more consistency\footnote{E.g. when an editor notices that an info box supports the parameter ``location'' but does not have one yet, the user may feel encouraged to complete it.}.
More importantly, the usage of a template in an article is lets that article become a member of the group of articles that embed this template.
This is an alternative mechanism to group articles that is likely to yield more homogenous results than the category system, see \nameref{sub:categories}.

\subsection{Revision history}\label{sub:revisionhistory}

Each submission of an edit in the browser creates a new revision of the article and is stored in the revision history, \imgref{history}.
Naturally, each article available today started from an empty page and is the result of a succession of edits.

\img{history}{Revision history of the article \emph{2011-2012 Bahraini uprising} on \dformat{2012}{01}{23}.}

Each entry in the revision history consists of the new wikitext, the date of submission, the user and an optional comment explaining the change.
In addition, authors have to possibility to mark an edit as \emph{minor}.
In doing so, they suggest that the submitted change is only superficial, e.g. correcting typography or moving passages of text, and does not change the meaning of the article.\cite{wpminor}

Each revision can not only be examined by other users and but also be reverted. 
Especially in the case of vandalism this mechanism can be used to restore the previous state of the article.
Reverts may also appear when different views on the same topic collide.
To minimize the potential for \term{edit wars}\cite{suh2007us} Wikipedia urges its users to discuss controversial topics on the article's talk page.


\subsection{Authors}\label{sub:authors}

Contributions to an article can be done anonymously or as a registered user.
A registered user gains privileges like the ability to create articles or the use of the social network features in Wikipedia.
With the initial registration a \emph{user page} is created where the user is allowed to publish a profile and interact with other registered users.\cite{wikiwhyaccount}
The majority of edits comes from registered users, anonymous edits account for a quarter of all edits.\cite{wpanonstats}

A third group of editors are automatic programs known as \emph{bots}.
They perform routine tasks ranging from spell-checking over curse word detection to automatic reverts on vandalism.
Currently the English Wikipedia alone has nearly approved 1,500 bot tasks running, either automatically or manually triggered by a real user.\cite{wpbots}

\subsection{User pages}\label{sub:userpages}

When a Wikipedia user decides to register, a \emph{user page} is created for him on Wikipedia's website. 
This is a special page that can be edited like any other article.
The user can publish personal information either in prose or reuse a template (see \nameref{sub:templates}).
The templates available to decorate one's user page include the following: 
\begin{itemize}
  \item Spoken languages, e.g. ``This user is a native speaker of English.''\footurl{http://en.wikipedia.org/wiki/Wikipedia:Babel}{2012}{01}{23}
  \item Location, e.g. ``This user comes from India.''\footurl{http://en.wikipedia.org/wiki/Template:User_India}{2012}{01}{23}
  \item Expression of personal views, e.g. ``This user opposes Imperialism.''\footurl{http://en.wikipedia.org/wiki/User:Serouj/UserBox/Against_Imperialism}{2012}{01}{23}
\end{itemize}
Of course, a user can publish just about anything.
As a result, the information on a user page has to be taken with a grain of salt.\footnote{See user Lihaas, who seems to hail both from India and Pakistan: \weburl{http://en.wikipedia.org/wiki/User:Lihaas}{2012}{01}{23}}

Like any other article, each user page has a discussion page that can be used to communicate with that user by leaving a message.
According to \textcite{Viegas2007talk} these pages ``hold much of the community interaction''.


\section{Contributions}\label{sec:contribution}

Wikipedia's articles are continuously edited by its users.
The nature of an edit can range from simple spelling or grammar correction, over improving the content of a sentence to writing or removing of paragraphs or even whole articles. 
This collective authorship makes it difficult to determine an individual author's contributions, in other words, it is not easy to tell who wrote what.

Research in this area tends to be motivated by the desire to identify individual authors with a good reputation in order to assign a trust score to them.
This is based on the assumption that trusted authors consistently produce high quality contributions that outlive contributions of lower quality.
\textcite{kramer2008wiki} devised a method to assign trust scores to the authors of an article by examining the wealth of information contained in the article's revision history.
They looked at an article as being a set of phrases.
The author who first wrote a sentence gets the credit for that phrase and will gain trust if it survives future edits.\footnote{\textcite{kramer2008wiki} define a sentence as an n-gram---a sequence of n words---and use a sliding window model to follow it across revisions to prevent simple rearrangements of text from counting as a new sentence.}

A similar approach of calculating the longevity of text chunks was followed by \textcite{adler2007content}.
They adapted standard text-diff algorithms to the peculiarities of the wiki revision system, e.g. keeping track of text chunks that were removed at one point and then reinserted in later revisions.
Based on these algorithms a reputation system was implemented by \textcite{adler2008assigning} which offers an API\footurl{http://www.wikitrust.net/vandalism-api}{2011}{10}{31} that can annotate a Wikipedia article.
The annotated text is the result of splitting the original text into chunks and attributing them with their respective authors, the number of the revision where the chunk was added and a trust value for the author.\footnote{\citeauthor{adler2008assigning} also released a Firefox add-on that highlights untrustworthy passages when viewing Wikipedia articles: \weburl{https://addons.mozilla.org/en-US/firefox/addon/wikitrust/}{2011}{11}{15}.}


\section{Georeferences}\label{sec:georeference}

In order to analyze the localness of contributions, it is necessary to geotag them, i.e. applying geospatial metadata like coordinates to each contribution, derived from the author's location.
In his doctoral thesis \textcite{hardy2011volunteered} used the Wikipedia corpora to study the spatial behavior of article production.
The dataset was limited to anonymous users and articles that were geotagged.

For each anonymous contribution an IP address, belonging to the point of Internet access, is stored in the revision that is created.
Various methods to determine the geographic location from a given IP address have been studied by \textcite{muir2009internet}.
Various visualizations\footurl{http://infodisiac.com/blog/2011/05/wikipedia-edits-visualized/}{2011}{10}{31} \footurl{http://sonetlab.fbk.eu/wikitrip/}{2011}{10}{31} of edit distributions use geolocation databases like MaxMind\footurl{http://www.maxmind.com}{2011}{10}{31} and Quova\footurl{http://www.quova.com}{2011}{10}{31}.

For registered users, the IP address is not stored with the revision.
Therefor IP geolocation services cannot be used.
\textcite{lieberman2009you} found an interesting approach by assuming users prefer to edit geographic articles in their proximity.
The approximated user location was derived from the center of the convex hull around those articles.

Registered users are also given the opportunity to create a personal profile in their \emph{user page}.
The user can choose prose or structured boxes to reveal information like his general interests, spoken languages, but also his location.
When entity names can be extracted from location information they can lead to coordinates as shown by \textcite{hecht2010localness}.