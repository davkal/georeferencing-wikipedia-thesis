%************************************************
\chapter{Foundation}\label{ch:foundation}
%************************************************

Wikipedia is a phenomenon that has attracted researchers across all fields, notably computer science and sociology, who have written over 1,000 reports on the subject to date.
\textcite{nielsen2011wikipedia} compiled an overview of Wikipedia research\footnote{Another resource is the Wikimedia Foundation's own directory of Wikipedia research projects at \weburl{http://meta.wikimedia.org/wiki/Research:Projects}{2011}{10}{12}.} and divides these publications into four categories: 
\begin{description}
\item[Content production] Covering all aspects of voluntary production such as motivation, collaboration, coverage and bias, quality and vandalism, actuality and geography.
\item[Information use] Treating how the resulting corpus is being used, e.g. Wikipedia citation in research, use in court, trend-spotting, natural language processing and automatic translation tools, thesaurus construction or categorization.
\item[Improvement] These are studies concerned with the improvement of both the software used by Wikipedia and the content, e.g. automatic linking, improved editors as well as quality and trust indicators. 
\item[Communication] Studies in this category look at Wikipedia as an online collaboration tool for education and research. 
\end{description}

This thesis falls into the first category, content production, as it examines the geography of article contributions that will become part of the Wikipedia's corpus.
After a short overview of Wikipedia from a user's perspective, I will introduce its model of collective authorship and present prior research of concerning location and geography. 

\section{Wikipedia}\label{sec:wikipedia}

Wikipedia is an online encyclopedia with editions in over 260 languages.
Counting 3.6 million articles, the English version is by far the biggest.
However, other language editions differ sharply in size and usage.\cite{wikistats}
If articles covering the same topic exist in other language editions, these are connected by interwiki links.

\subsection{History \cite{wphistory}}

Wikipedia was officially started on 15 January 2001 by Jimmy Wales and Larry Sanger.
Wales previously founded Nupedia, a free and peer-reviewed online encyclopedia written only by experts.
However, the speed of content production was extremely low.
Wikipedia was founded as a feeder project to collectively write on articles before these entered Nupedia's review process.
Wiki\-pedia then quickly created other language editions and dwarfed its predecessor\footnote{Only 24 articles were completed in Nupedia's review process. The project was officially ended in 2003.}.

After being mentioned on Slashdot, a technology news website, in March 2001 Wikipedia quickly attracted new users.
This tech-savvy folk created new articles at a staggering rate of 1,500 articles per month in the first year.
These articles then quickly started showing up in Google's search results, attracting even more new users.
The non-English editions grew slower but as a group accounted for 75\% of all articles in 2007.
By 2011 the combined article count passed 20 million.

\subsection{Wikimedia Foundation}

\todo{Foundation structure, independence of language wikis, funding}

\subsection{Anatomy of an article}

\todo{Image: graphic of article UI}

\todo{describe article anatomy, info boxes with dates and geotags}

Some articles contain geographic coordinates as part of the content, e.g. the article about the \term{Brandenburg Gate}\footurl{http://en.wikipedia.org/wiki/Brandenburg_Gate}{2011}{10}{31} in Berlin is tagged with the coordinates 52°30'58.58''N 13°22'39.80''E.
However, these articles are geographic in nature treating cities, rivers and places of interest. 
This thesis hopes to expand the spatial analysis of contributions to articles that have a location property only by association, e.g. the \term{Egyptian Revolution of 2001}\footurl{http://en.wikipedia.org/wiki/Egyptian_Revolution_of_2011}{2011}{10}{31}, that clearly happened in \term{Egypt}\footurl{http://en.wikipedia.org/wiki/Egypt}{2011}{10}{31}.


\subsection{Wikipedia, wikis and editing}

Anyone with a browser and internet access can edit Wikipedia's articles\footnote{Some articles can be locked because of sustained vandalism or content disputes.\cite{wpprotectionpolicy}}.
In collaboration, people all over the world contribute and improve the content. 
This is made possible by MediaWiki, the the server software that makes Wikipedia a wiki. 
The software allows website visitors to add and modify the page content in the browser using \emph{wikitext}, simplified markup language.\footnote{For the syntax see \weburl{http://en.wikipedia.org/wiki/Help:Wiki_markup}{2011}{12}{12}.}
Its syntax can be used to structure a text into sections, embed images and links to other pages, much like HTML.
The syntax is explicitly kept simple to keep the entry barrier to editing low.

Each submission of an edit in the browser creates a new revision of the article and is stored in the revision history.
Naturally, each article available today started from an empty page and is the result of a succession of edits.
Each entry in the revision history consists of the new wikitext, the date of submission, the user and an optional comment explaining the change.
Each revision can not only be examined by other users and but also reverted.
To minimize the potential for \term{edit wars}\cite{suh2007us} Wikipedia urges its users to discuss controversial topics on the article's talk page.

\todo{Image: revision history}

Contributions to an article can be done anonymously or as a registered user.
A registered user gains privileges like the ability to create articles or the use of the social network features in Wikipedia.
With the initial registration a \emph{user page} is created where the user is allowed to publish a profile and interact with other registered users.\cite{wikiwhyaccount}
The majority of edits comes from registered users, anonymous edits account for a quarter of all edits.\cite{wpanonstats}

A third group of editors are automatic programs known as \emph{bots}.
They perform routine tasks ranging from spell-checking over curse word detection to automatic reverts on vandalism.
Currently the English Wikipedia alone has nearly approved 1,500 bot tasks running.\cite{wpbots}

\subsection{User pages}

\todo{write about user pages, prose, boxes}

\section{Contributions}\label{sec:contribution}

Wikipedia's articles are continuously edited by its users.
The nature of an edit can range from simple spelling or grammar correction, over improving the content of a sentence to writing or removing of paragraphs or even whole articles. 
This collective authorship makes it difficult to determine an individual author's contributions, in other words, it is not easy to tell who wrote what.

Research in this area tends to be motivated by the desire to identify individual authors with a good reputation in order to assign a trust score to them.
This is based on the assumption that trusted authors consistently produce high quality contributions that outlive contributions of lower quality.
\textcite{kramer2008wiki} devised a method to assign trust scores to the authors of an article by examining the wealth of information contained in the article's revision history.
They looked at an article as being a set of phrases.
The author who first wrote a sentence gets the credit for that phrase and will gain trust if it survives future edits.\footnote{\textcite{kramer2008wiki} define a sentence as an n-gram---a sequence of n words---and use a sliding window model to follow it across revisions to prevent simple rearrangements of text from counting as a new sentence.}

A similar approach of calculating the longevity of text chunks was followed by \textcite{adler2007content}.
They adapted standard text-diff algorithms to the peculiarities of the wiki revision system, e.g. keeping track of text chunks that were removed at one point and then reinserted in later revisions.
Based on these algorithms a reputation system was implemented by \textcite{adler2008assigning} which offers an API\footurl{http://www.wikitrust.net/vandalism-api}{2011}{10}{31} that can annotate a Wikipedia article.
The annotated text is the result of splitting the original text into chunks and attributing them with their respective authors, the number of the revision where the chunk was added and a trust value for the author.\footnote{\citeauthor{adler2008assigning} also released a Firefox add-on that highlights untrustworthy passages when viewing Wikipedia articles: \weburl{https://addons.mozilla.org/en-US/firefox/addon/wikitrust/}{2011}{11}{15}.}


\section{Georeferences}\label{sec:georeference}

\todo{''Combined approaches (i.e., where quantitative spatial analysis models are calibrated with surveyed locations) may prove useful.''  \cite[p. 85]{hardy2011volunteered}}

In order to analyze the localness of contributions, it is necessary to geotag them, i.e. applying geospatial metadata like coordinates to each contribution, derived from the author's location.
In his doctoral thesis \textcite{hardy2011volunteered} used the Wikipedia corpora to study the spatial behavior of article production.
The dataset was limited to anonymous users and articles that were geotagged.

For each anonymous contribution an IP address, belonging to the point of Internet access, is stored in the revision that is created.
Various methods to determine the geographic location from a given IP address have been studied by \textcite{muir2009internet}.
Various visualizations\footurl{http://infodisiac.com/blog/2011/05/wikipedia-edits-visualized/}{2011}{10}{31} \footurl{http://sonetlab.fbk.eu/wikitrip/}{2011}{10}{31} of edit distributions use geolocation databases like MaxMind\footurl{http://www.maxmind.com}{2011}{10}{31} and Quova\footurl{http://www.quova.com}{2011}{10}{31}.

For registered users, the IP address is not stored with the revision.
Therefor IP geolocation services cannot be used.
\textcite{lieberman2009you} found an interesting approach by assuming users prefer to edit geographic articles in their proximity.
The approximated user location was derived from the center of the convex hull around those articles.

Registered users are also given the opportunity to create a personal profile in their \emph{user page}.
The user can choose prose or structured boxes to reveal information like his general interests, spoken languages, but also his location.
When entity names can be extracted from location information they can lead to coordinates as shown by \textcite{hecht2010localness}.