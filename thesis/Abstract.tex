%*******************************************************
% Abstract
%*******************************************************
%\renewcommand{\abstractname}{Abstract}
\pdfbookmark[1]{Abstract}{Abstract}
\begingroup
\let\clearpage\relax
\let\cleardoublepage\relax
\let\cleardoublepage\relax

\chapter*{Abstract}

Wikipedia is more than an online encyclopedia. 
It is also a news channel as well as a self-updating history book.
A global readership can follow political events as they unfold, written about by local people and later edited by other volunteers.
This thesis describes a method to answer the question to what extent local volunteers write about events in their own country.
First, the geographic origin of each individual article contribution is determined.
In a second step, a given article is annotated with georeferences on a word level.
The properties of these annotations then allow for a statistical geographic analysis of a single article or a category of articles.

\vfill

\pdfbookmark[1]{Zusammenfassung}{Zusammenfassung}
\chapter*{Zusammenfassung}

\todo{Kurze Zusammenfassung des Inhaltes in deutscher Sprache, behandle Forschungsfrage, Weg zur Beantwortung und Ergebnis in allgemeinverständlicher Sprache\dots}

Als Online-Enzyklopädie ist Wikipedia nicht nur Nachschlagewerk sondern auch 
ein sich stetig wandelndes Geschichtsbuch. Eine global verteilte Nutzerschaft liest 
und schreibt über lokale Ereignisse noch während sie passieren. Diese Arbeit soll
Möglichkeiten untersuchen, inwiefern man die Herkunft der Autoren bestimmen
und damit Einflusssphären auf politische Ereignisse sichtbar machen kann. 
Vorhandene Analysemethoden und Visualisierungen sollen auf Eignung untersucht, 
gegebenenfalls weiterentwickelt und als Proof of Concept in einer Software umgesetzt werden.

\endgroup			

\vfill