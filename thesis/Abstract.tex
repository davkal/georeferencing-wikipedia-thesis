%*******************************************************
% Abstract
%*******************************************************
%\renewcommand{\abstractname}{Abstract}
\pdfbookmark[1]{Abstract}{Abstract}
\begingroup
\let\clearpage\relax
\let\cleardoublepage\relax
\let\cleardoublepage\relax

\chapter*{Abstract}

Wikipedia is more than an online encyclopedia. 
It is also a news channel as well as a self-updating history book.
A global readership can follow political events as they unfold, written about by local people and later edited by other volunteers.
This thesis describes a method to answer the question to what extent local volunteers write about events in their own country.
First, the geographic origin of each individual article contribution is determined.
In a second step, a given article is annotated with georeferences on a word level.
The properties of these annotations then allow for a statistical geographic analysis of a single article or a category of articles.

\vfill

\pdfbookmark[1]{Zusammenfassung}{Zusammenfassung}
\chapter*{Zusammenfassung}

Als Online-Enzyklopädie ist Wikipedia nicht nur Nachschlagewerk sondern auch ein sich stetig wandelndes Geschichtsbuch. 
Eine global verteilte Nutzerschaft liest und schreibt über lokale Ereignisse noch während sie passieren. 
Diese Arbeit beschreibt eine Methode zur Bestimmung des Anteils an Beitr\"agen, die vom betreffenden Land ausgehen.
In einem ersten Schritt werden die geographischen Urspr\"unge aller Beitr\"age eines Artikels ermittelt.
Mit den daraus erhaltenen Georeferenzen wird der Artikel Wort f\"ur Wort annotiert.
Basierend auf diesen Annotationen kann dann der lokale Autoren-Anteil bestimmt werden.
\endgroup			

\vfill