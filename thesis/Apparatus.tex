%************************************************
\chapter{Apparatus}\label{ch:apparatus}
%************************************************

This chapter describes data sources to get Wikipedia content like articles and revision history as well as tools to retrieve and analyze those.

\todo{Intro aspects of apparatus}

\section{Data sources}

For an automated analysis, simply browsing Wikipedia's website is not really feasible. 
The bulk of Wikipedia's content like articles, revisions, discussions is stored on its database servers.
Unfortunately, these databases are not directly accessible over the Internet.
The Wikimedia Foundation, however, makes a lot of the data available in the form of database dumps or through an application programming interface (API).

\subsection{Wikipedia website}

For a complete article analysis, navigating the website can be tedious as one would have to click through a complete revision history and parse the page's source which is formatted using the HyperText Markup Language (HTML).
However, individual pages contain data that is static and can be used throughout the analysis process.
This makes it worth writing a specific parser for a technique known as \emph{screen scraping} to extract the information.
On a high level, it involves the following steps:

\begin{enumerate}
  \item Looking at the HTML source of the page and identifying how the HTML tags and attributes that are used to structure the information.
  \item Writing a parser that addresses the identifying tags and thereby tokenizes the data.
  \item Converting the found tokens into an output format, e.g. JSON.
\end{enumerate}

For a simple HTML table, a parser can be written in a few lines of code.
Using this technique, the following static information was gathered:

\begin{description}
\item[bots] A list of bots was built based on the Wikipedia page \emph{List of bots by number of edits}\footurl{http://en.wikipedia.org/wiki/Wikipedia:List_of_bots_by_number_of_edits}{2012}{01}{24}. 
This list is used to distinguish bots from real authors as contributions done by bots are excluded from the analysis.
There are unregistered bots, however, that appear not in the list.
For a lack of automated distinction, these are counted as normal authors.\footnote{A simple heuristic employed by other software to analyze MediaWiki content is treating all contributors whose username contains or whose comments start with ``bot'' as a bot, e.g. pymwdat, see \weburl{http://code.google.com/p/pymwdat/source/browse/trunk/toolkit.py?spec=svn13&r=13}{2012}{01}{24}. This has a potential for false positives and is not used.}
\item[countries] A list of countries was extracted from the article \emph{ISO\_3166-1}\footurl{http://en.wikipedia.org/wiki/ISO_3166-1}{2012}{01}{2}. 
It provides a list of standardized country names that is also respected by Wikipedia's authors when referring to a country by name.
In a second pass, the Wikipedia article of each country was parsed for coordinates.\footnote{For some countries, coordinates were not present on the page, e.g. \weburl{http://en.wikipedia.org/wiki/Australia}{2012}{01}{24}. In that case, they were manually added by using that country capital's coordinates. For a discussion on geographic resolution, see \nameref{sub:resolution}.}
\end{description}

Making both of these sets static was a design decision recognizing the trade-off between having them in memory and querying for each article.

\subsection{Database dumps}\label{sub:dumps}

Monthly database snapshots of all wikis run by the Wikimedia Foundation, including Wikipedia,  are publicly available\footurl{http://dumps.wikimedia.org}{2011}{12}{11} as database dump files in the XML file format.
For each of the wikis a variety of dumps is available that include all articles and, optionally, their revision history, all categories, interwiki-links, etc.
Despite this openness, some database tables are not publicly available.
The dump files of the database tables \emph{users} and the \emph{watchlist} are kept private.

The dump files can be quite large, e.g. a compressed dump of all articles of the English Wikipedia in their current revision has a size 7.3 GB.\footnote{The uncompressed size is 31.0 GB, see \weburl{http://en.wikipedia.org/wiki/Wikipedia:Database_download}{2011}{12}{11}}
This huge size makes processing them rather slow.\footnote{The project WikiHadoop addresses this problem by offering a stream task format to be used in Hadoop (MapReduce) infrastructure, see \weburl{https://github.com/whym/wikihadoop}{2011}{12}{11}.}
When analyzing only a single article or a category articles, the MediaWiki API can deliver the same information contained in the dumps in a much more targeted manner. 

\subsection{MediaWiki API}\label{sub:mediawikiapi}

Wikipedia runs on the open source software MediaWiki.
This PHP-based wiki package offers a well documented API\footurl{http://www.mediawiki.org/wiki/API:Main_page}{2012}{01}{24} which can be used by other programs to remotely use the wiki's features such as changing content and restoring revisions.\footnote{The full capability of the API can be seen at tried at the \emph{Sandbox} at \weburl{https://en.wikipedia.org/wiki/Special:ApiSandbox}{2012}{01}{24}, a recent addition to the MediaWiki software.}
For analysis of articles, the API offers queries directed at a variety of article properties, e.g. revisions, categories and links.
Among the output formats for the responses is the JavaScript Object Notation (JSON).
Similar to MediaWiki's Special:Export page\footnote{The page \weburl{https://en.wikipedia.org/wiki/Special:Export}{2011}{12}{11} allows for exporting of articles from the English Wikipedia.}, the API also offers an article export that includes all revisions.
\lstset{caption={Example JSON response to a query to list all bots that edited the article \emph{2011-2012 Bahraini uprising}},label=apicall}
\begin{lstlisting}

	{
		"query": {
			"redirects": [{
					"from": "2011 Bahraini uprising",
					"to": "2011-2012 Bahraini uprising"
			}],
			"pages": {
				"30876395": {
					"pageid": 30876395,
					"ns": 0,
					"title": "2011-2012 Bahraini uprising"
				}
			},
			"allusers": [{
					"userid": "13146235", "name": "28bot"
				}, {
					"userid": "5415725", "name": "718 Bot"
				}, ..., {
					"userid": "13770078", "name": "AWBCPBot"
			}]
		},
		"query-continue": {
			"allusers": {
				"aufrom": "AWeenieBot"
			}
		}
	}
\end{lstlisting}

Some of the queries have a limit on how many results they return on a single request.
When there are more results, the response contains a \emph{query-continue} attribute that can be sent with following query so that the next result set can be returned.
The following API calls will be important for this thesis:

\begin{description}
    \item[query info] This basic query is returns essential information like the article ID, the last revision ID, but also the full wikitext of the last revision.
    \item[query revisions] Lists all revisions for an article and for each includes a timestamp and the user as well as the comment for the text change.
    \item[query categorymembers] For a given category, this query lists the articles and subcategories belong to it.
    This query can be used to construct groups of articles for analysis in this thesis.
    \item[query templateembedders] For a given template name, this request lists all pages that embed it.
    This query can also be used to build a group of articles. 
    \item[open search] A method to suggest articles, categories and templates that contain a term. 
    It can be attached to an input field where a user is supposed to enter the name of an article.
    \item[parse] This special query returns the HTML version of the article's wikitext.
    The content that is returned is exactly the HTML source that is sent to a browser when a visitor looks at this article or user page.
    This query will be used in cases where it is easier to parse the HTML markup than the wikitext, e.g. the \nameref{sub:userpages}.
\end{description}

\subsection{Toolserver}\label{sub:toolserver}

The Germany based Wikimedia Deutschland e.V. runs Toolserver\footurl{http://toolserver.org}{2011}{12}{11}, a platform for software tools that can access a continuously updated copy of Wikipedia's databases. 
Among these replicated databases is the English Wikipedia and other major language editions.
However, the deployment of self-made software scripts is restricted and requires an account on Wikimedia's Toolserver.\footnote{I applied for a Toolserver account outlining my necessary database queries, usage profile as well as my affiliation with the Freie Universit\"at Berlin. The application was submitted on 2011-12-21 and has not been processed yet (2012-01-23).}

Some scripts that are already deployed can be accessed freely, allowing them to be reused.
One of these was developed by SoNet\footurl{http://sonetlab.fbk.eu/}{2011}{12}{12}, a social networking research group based in Italy, for a project called WikiTrip (see \nameref{sub:analysisprojects}).
It offers an API\footnote{The API is documented here: \weburl{https://github.com/volpino/toolserver-scripts/tree/master/php}{2011}{12}{12}} to get simple article statistics like article ID, text length as well as complex data structures like a list of unique editors including their gender if they are registered users and chose to reveal their gender in their Wikipedia account.\footnote{Try \weburl{http://toolserver.org/~sonet/api_gender.php?article=Egypt&lang=en}{2011}{12}{11} to get a list of all registered users who edited the article \emph{Egypt} of the English Wikipedia.}

In effect, calling the SoNet API replaces several calls to the original MediaWiki API and therefor speeds up the information retrieval, especially when the number of revisions or authors is high.
The returned data object has the following structure:

\lstset{caption={SoNet API response to a query for the article \emph{2011-2012 Bahraini uprising}},label=sonetapicall}
\begin{lstlisting}
	{
		"first_edit": {"timestamp":1297734917,"user":"Master&Expert"},
		"count":1778,
		"minor_count":401,
		"count_history":{"today":3,"week":5,"month":90,"year":1778},
		"last_edit":1327370324,
		"totaldays":0,
		"average_days_per_edit":"0.00",
		"edits_per_month":0,
		"edits_per_year":0,
		"edits_per_editor":"4.17",
		"editor_count":426,
		"anon_count":337,
		"editors": {"Bahraini Activist":
			{"all":106,"minor":21,"first":"17 May 2011, 09:45:25",
			"last":"22 January 2012, 10:52:50","atbe" 203811,
			"minorpct":"19.81", "size":"140.54","urlencoded":"Bahraini_Activist"},
			...
		},
		"anons":{"2011-02-15T08:11:52Z":
			["78.2.29.139","Rovinj Croatia",45.08,13.64],
			...
		}
	}
\end{lstlisting}

This high density of preprocessed information shows the power of the Toolserver and its direct access to the database.
The property \emph{editors} lists all unique authors of an article and their edit count (property \emph{all}).
The second exhaustive collection is under the property \emph{anons}. 
There, all anonymous contributors are listed with their IP address, their geographic region and coordinates.
The geographic lookup is uses\footurl{https://github.com/volpino/toolserver-scripts/blob/master/php/api.php}{2012}{01}{24} the GeoCityLite database from Maxmind (see \nameref{sub:iplookup} for a discussion).

\subsection{Third-party sources/Web services}\label{sub:webservices}

Like the Toolserver scripts in the previous section, other research projects exist that can be reused as data sources.
Depending on the project's goal, a variety of preprocessed data is available:

\begin{description}
\item[Article traffic] Wikipedia user Henrik\footurl{http://en.wikipedia.org/wiki/User:Henrik}{2011}{12}{12} provides a web service that processes Wikipedia's log files\footnote{These are available at \weburl{http://dumps.wikimedia.org/other/pagecounts-raw/}{2011}{12}{12}} to calculate the number page views per article for a given time.
These statistics can be viewed through a browser\footnote{E.g. \weburl{http://stats.grok.se/en/201105/2011_Egyptian_Revolution}{2011}{12}{12}} or queried through an API \footnote{E.g. \weburl{http://stats.grok.se/json/en/201105/2011_Egyptian_Revolution}{2011}{12}{12}}. 
\item[CatScan] This web service, offered by Toolserver administrator Duesentrieb\footurl{http://meta.wikimedia.org/wiki/User:Duesentrieb}{2011}{12}{13}, finds articles that belong to a given category and its sub-categories (see \nameref{sub:categories} on why this is non-trivial).
It also offers to limit the search to an intersection of categories, e.g. German politicians who are also physicists\footurl{https://toolserver.org/~daniel/WikiSense/CategoryIntersect.php?wikilang=de&wikifam=.wikipedia.org&basecat=Politiker+(Deutschland)&basedeep=3&mode=cs&tagcat=Physiker&tagdeep=2}{2011}{12}{13}.
The results are presented in the browser or can be downloaded as a file containing comma-separated values (CSV format). 
\item[Poor man's checkuser] The project \emph{Poor Man's Check User}\footnote{Project website: \weburl{http://wikiwatcher.virgil.gr/pmcu}{2012}{01}{2}. The project's name is a reference to the \emph{checkuser} permission that a community-elected group of registered users possesses. It allows a de-masking of the IP addresses for each of a registered user's edit.} mapped registered users to IP addresses based on a bug in the session management of the MediaWiki software.\footnote{When a user exceeded a certain time while editing an article without submitting the current changes, the user's session expired on the server. When the edit was submitted after the expiration the user appeared as an anonymous author, being only known by his IP address. When the user then logged in again, the same change was sent again. Scanning all revisions for the same change set therefor allowed for a matching between user name and IP address. This loophole has been closed, however.} 
For the period the bug has been active, some usernames could be mapped.
Naturally, the more edits a user did in this period, the more likely is an appearance in this list.
For the purpose of this thesis, I screen-scraped this table and condensed\footnote{Some usernames have multiple entries as each occurrence of the bug created a unique ``evidence''. Among those, some have been manually verified and ranked. When multiple entries exist, my algorithm picks the top ranked.} it to 14,171 unique users.
\item[Quova] This geo-location web service maps an IP address to a geographic location, see \nameref{sec:georeference}.
\item[WikiTrust] Based on \textcite{adler2008assigning} an open source online reputation system\footurl{http://www.wikitrust.net/vandalism-api}{2011}{10}{31} was set up by the University of California, Santa Cruz, to allow for easy vandalism detection (see \nameref{sec:contribution}).
Given an article ID and a revision ID, the API method \emph{wikimarkup} returns an annotated version of the wikitext of that revision.
An annotation consists of a trust value, the revision ID the text got introduced into the article as well as the authors user name or IP address, e.g. revision 473029564 of the article \emph{2011-2012 Bahraini uprising}\footurl{http://en.collaborativetrust.com/WikiTrust/RemoteAPI?method=wikimarkup&pageid=30876395&revid=473029564}{2012}{01}{23}:
\lstset{language=HTML,caption={Excerpt of the annotated markup for the revision \\473029564 of the article \emph{2011-2012 Bahraini uprising}},label=wikitrustapicall}
\begin{lstlisting}
	{{#t:7,468889105,Kudzu1}}The 
	{{#t:7,470041169,Happysailor}}2011-2012 
	{{#t:8,413989516,Master&Expert}}Bahraini 
	{{#t:8,427545590,Kudzu1}}uprising, sometimes called the
	{{#t:9,455029613,Sitrawi86}}February 14 Revolution 
\end{lstlisting}
All wikitext following an annotation, up to the next one, was written by that author.
The web service provider implemented a custom diff algorithm for the attribution of authorship.
This was needed to overcome wiki-specific issues and to maximize tracking, e.g. for text that is removed and re-inserted at a later revision.\footurl{http://www.wikitrust.net/frequently-asked-questions-faq\#TOC-On-text-author-and-origin}{2012}{01}{24}
\end{description}


\section{Available Tools}

To process the data from all the data sources, a wide range of software tools are available in the open source community.
A simple search for ``Wikipedia'' on GitHub\footurl{https://github.com/search?q=wikipedia&type=Repositories}{2011}{12}{12}, a source code exchange platform, shows a multitude of small software projects.
These come in different programming languages and different feature sets and usually help in downloading articles in batches and extract data from big dump files. 
Developed by vigilantes and researchers alike, these programs facilitate both data retrieval and processing.

\subsection{Toolkits}\label{sub:toolkits}

A group of openly available software packages\footnote{Although none of these were used for the content analysis of this thesis, their study proved very insightful on how to process Wikipedia's content.} qualify as swiss-army knifes for processing and analyzing \nameref{sub:dumps}:

\begin{description}
\item[pywikipedia] As the mother of all Python toolkits, the Python wikipedia robot framework\footurl{http://pywikipediabot.sourceforge.net/}{2012}{01}{24} offers an extendable set of classes for all MediaWiki entities like a page, a user, revision, etc, and is typically used to write a bot program for automated editing tasks (see \nameref{sub:authors}).
\item[pymwdat] Based on \entity{pywikipedia}, this toolkit offers a convenient downloader for all revisions of an article as well as an extensible dump file analyzer with support for filtering and revert detection.
\item[levitation] A creative project to turn \nameref{sub:dumps} into a Git\footurl{http://git-scm.com/}{2012}{01}{24} repository.
As a source code management system, Git offers a more space efficient way to store the sequence of revisions of the articles, since it only stores the difference between revisions.
Once converted to a repository, moving from revision to revision is much faster than processing the large dump files.
Git's diff mechanism together with its \emph{blame} command can be used as an alternative way to attribute authorship to passages of article content.\footnote{In fact, git operates on a line level, making the attribution rather coarse. To get blaming functionality on a word level, I patched the source, see my fork at \weburl{https://github.com/davkal/levitation/commit/5fca0001d26cb67fde6ff9d8a5f2b1414cf7681e}{2012}{01}{24}.}
\end{description}

\subsection{Analysis projects}\label{sub:analysisprojects}

In addition to the toolkits, a handful of research projects exist that process Wikipedia's content.
These are purpose built applications that have a much narrower focus but are very skillful in combining and using different data sources such as \nameref{sub:mediawikiapi}, the \nameref{sub:toolserver} or other \nameref{sub:webservices}:

\begin{description}
\item[WikiPride] Python web-application\footurl{https://github.com/declerambaul/WikiPride}{2011}{12}{11}  to visualize contributions of groups of editors that registered in the same month.\footnote{Project website: \weburl{http://meta.wikimedia.org/wiki/Research:WikiPride}{2011}{12}{11}}
\item[Wiki Trip] JavaScript application\footurl{https://github.com/volpino/wikipedia-timeline}{2011}{12}{11}, written at SoNet, that uses the \nameref{sub:mediawikiapi} as well as its own \nameref{sub:toolserver} scripts, to visualize the evolution of a single article over time including: anonymous vs. registered contributors, male vs. female registered users, anonymous edits by country.\footnote{Live demo: \weburl{http://sonetlab.fbk.eu/wikitrip/}{2011}{12}{11}} 
\end{description}


\section{Application design}

\todo{following wikitrip design of a browser application, based on the following technologies}

\subsection{Technologies}

\subsection{Models}

\subsection{Views}

\subsection{Operation}

\begin{todos}
    \item fetch article, wikitext and HTML
    \item parse for dates and location (delegate to subpage) and evaluate article relevancy
    \item fetch authors (toolserver)
    \item fetch revisions
    \item fetch first revision of each language
    \item fetch page views
    \item locate users
    \item fetch annotated source from WikiTrust for selected revisions 
    \item render/store results
\end{todos}

\todo{group mode}

\begin{todos}
    \item fetch article list
    \item analyze all articles
    \item render group results
\end{todos}


\section{Algorithms}

\todo{intro, complementary HTML and wikitext, HTML was easier offering annotations like dtstart}

\subsection{Article requirements}

\todo{list requirements}

\subsection{Date parsing}

\todo{link to Template}

% Tricky http://en.wikipedia.org/wiki/2007_Georgian_demonstrations
% The demonstrations peaked on November 2, 2007, ...

\subsection{Location parsing}

\todo{link to Template}

\subsection{Collective authorship}

\todo{Introduce types of authors (roles) as well as methods to determine contribution/attribution}

\begin{todos}
    \item Autoren
    \item Bots
    \item text attribution, using wikitrust annotations, not for all revisions
\end{todos}

\todo{Are all edits relevant? Edit wars? Bots?}


\subsection{Resolving user names to IPs}

\begin{todos}
    \item registered vs. unregistered vs. bots vs. admins
    \item incorporate key findings of \cite{hardy2011volunteered} as laid out in chapter \ref{sec:georeference}
    \item IPs of unregistered users: Geo lookup
    \item Autoren-Profile: Information Extraction
    \item Geographische Zuordnung vom user profile
\end{todos}

%Zur Bestimmung der Herkunft eines Autors bietet Wikipedia zwei direkte Ansätze: 
%Für jeden Beitrag eines nicht registrierten Benutzers wird die IP-Adresse gespeichert, über die er Zugang zum Internet erlangt hat. 
%Der zweite Ansatz betrifft die registrierten Benutzer.
%Ihre IP-Adressen sind maskiert und nicht öffentlich zugänglich.
%\footnote{Eine kleine, von der Wikipedia-Community gewählte Nutzerschaft mit der Berechtigung \emph{checkuser} kann die Adressen demaskieren.}


\subsection{IP Look-up}\label{sub:iplookup}

\begin{todos}
    \item Services
    \item Accuracy
    \item Active prevention by proxies and anonymizers: 
    \\ \fullcite{muir2006internet} 
    \\ \fullcite{muir2009internet}
    \\ \fullcite{duckham2005formal}
\end{todos}

%Mit frei verfügbaren\footnote{Die vorgestellten Dienste haben ein tägliches Kontingent an Anfragen. Hilfstechniken wie Caching können diese Einschränkungen jedoch mindern.} Online-Diensten wie \term{Quova}\footnote{\url{http://developer.quova.com}} oder \term{geoplugin}\footnote{\url{http://www.geoplugin.com/webservices}} lässt sich für einen Großteil der IPs daraufhin das Herkunftsland bestimmen.
%
%Im Bezug auf die Herkunft sind sowohl das Land als auch die Geo-Koordinaten interessant.  
%Basierend auf der Versionsgeschichte würde für nicht registrierte Benutzer eine Gewinnung von Daten dann beispielsweise folgende Schritte durchlaufen:
%
%\begin{quotation}
%IP \RA Geolocation-Dienst \RA Koordinaten und Land
%\end{quotation}


\subsection{Parsing user pages}

\begin{todos}
\item Automatic annotation of entities: \url{http://wikipedia-miner.cms.waikato.ac.nz/demos/annotate/}, also has services for categories. Alternative: \url{http://tagme.di.unipi.it/}
    \item IE approach with Machine Learning \fullcite{xiao2004information}
    \item unsupervised IE: \fullcite{etzioni2005unsupervised}
    \item if city is mentioned, determine country (needs disambiguation, e.g. Berlin)
    \item coordinates are optional?
\end{todos}

% IE algorithms based on Templates
%A(723) 755 -> 758 -> 802
%B(132) 143 -> 146 -> 161
%D(85) 91 -> 91 -> 99
%1. Country link part of RegExp, e.g. " comes? from Russia"
%2. Country part of RegExp (no link)
%3. Container around country link contained RegExp. e.g. "and live in a rather small town close to <a>my country's</a> capital"


\subsection{Geographic resolution}\label{sub:resolution}

\begin{todos}
    \item settle for a country
    \item some examples on accuracy for different countries
    \item clustering of origins: areas of influence
\end{todos}


\section{Visualization}

%Auf Basis der strukturierten Daten in Form von Artikeln, Sätzen, Ländern, Koordinaten und Sprachen sollen nun Visualisierungen gefunden werden, welche die Fülle an Informationen zugänglich machen.


\begin{todos}
    \item \fullcite{kjellin2010evaluating}
    \item Identify. Characteristics of an object.
    \item Locate. Absolute or relative position.
    \item Distinguish. Recognize as the same or different.
    \item Categorize. Classify according to some property (e.g., color, position, or shape).
    \item Cluster. Group same or related objects together.
    \item Distribution. Describe the overall pattern.
    \item Rank. Order objects of like types.
    \item Compare. Evaluate different objects with each other.
    \item Associate. Join in a relationship.
    \item Correlate. A direct connection.
\end{todos}

\subsection{Maps}

\begin{todos}
    \item Darstellung der geographischen Analyse
    \item per Wort, Satz, Artikel, Wort
\end{todos}

\subsection{Maps}

%\begin{labeling}{V2}
%\item[V1] Revisionshistogramm à la Google Finance 
%\item[V2] \emph{Heatmap} einer Landkarte mit Ursprüngen der Revisionen 
%\item[V3] Netzwerkgrafik, die Metriken desselben Artikels in verschiedenen Sprachvarianten anzeigt
%\item[V4] Dynamisches Blasendiagramm\footnote{\url{http://en.wikipedia.org/wiki/Motion_chart}} über die Entwicklung unterschiedlicher Sprachvarianten
%\item[V5] \emph{Heatmap} des Artikels mit Stellen höchster Aktivität 
%\item[V6] Landeskürzel für eine gegebene Textstelle
%\item[V7] Edit wars on map, linking two or more places
%\end{labeling}
%

\subsection{Line and scatter charts}


\subsection{Motion chart}


\section{Hypotheses analysis}

\todo{for each hypothesis, what data is gathers, how is it crunched}

\subsection*{H1}

\subsection*{H2}

\subsection*{H3}

\subsection*{H4}

\subsection*{H5}

\subsection*{H6}

\subsection*{H7}

\subsection*{H8}

\subsection*{H9}

\subsection*{H10}

\subsection*{H11}


\section{Possible enhancements}

\subsection{Edit relevance}

\todo{detect reverts, vandalism}

\subsection{Geographic profiling}

\begin{todos}
    \item \fullcite{lieberman2009you}
    \item \fullcite{hecht2010localness}
    \item from other fields such as criminal research: \\ \fullcite{snook2005complexity}
    \item feasibility, maybe just as enhancer
\end{todos}
