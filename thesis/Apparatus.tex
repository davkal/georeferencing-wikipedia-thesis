%************************************************
\chapter{Apparatus}\label{ch:apparatus}
%************************************************

Once the contributions have been georeferenced, the question of localness of the contributions can only be answered if the article itself can be geographically designated. 
A handful of studies exist, notably \cite{hardy2011volunteered,hecht2010localness}, that perform a spatial analysis on Wikipedia articles that are geo-tagged.
These articles contain geographic coordinates as part of the content, e.g. the article about the \term{Brandenburg Gate}\footurl{http://en.wikipedia.org/wiki/Brandenburg_Gate} in Berlin is tagged with the coordinates 52°30'58.58''N 13°22'39.80''E.
However, these articles are geographic in nature treating cities, rivers and places of interest. 
This thesis hopes to expand the spatial analysis of contributions to articles that have a location property only by association, e.g. the \term{Egyptian Revolution of 2001}\footurl{http://en.wikipedia.org/wiki/Egyptian_Revolution_of_2011}, that clearly happened in \term{Egypt}\footurl{http://en.wikipedia.org/wiki/Egypt}.


\todo{What tools do I have and how can they be extended.}

\section{Wikipedia's Data Structures}

\begin{description}
\item[Artikel] Ein Artikel hat mindestens einen Autor und ist gegebenenfalls in mehreren Sprachen vorhanden.
\item[Versionsgeschichte] Diese Historie liefert Informationen wie Benutzername oder IP-Adresse, Datum der Version sowie die inkrementelle Textänderung.
\item[user pages \& user boxes] Auf den \emph{user pages} kann ein registrierter Benutzer Informationen über sich veröffentlichen, die Aufschluss über seine Herkunft geben könnten.
\item[Externe Quellen] Im Internet existieren zahlreiche Dienste, die Schnittstellen anbieten, um Informationen über Nutzer und deren Beiträge zu erhalten, z.B.: WikiTrust\footref{wikitrust} oder WikiWatcher\footnote{Das WikiWatcher-Teilprojekt \emph{Poor Man's Check User} erlaubt eine Auflösung des Benutzernamens in eine IP-Adresse, wenn dieser Nutzer in der Vergangenheit beim Ändern eines Artikels das Session-Limit überschritten hatte. Inzwischen wurde diese Sicherheitslücke in der WikiMedia-Software jedoch behoben. \url{http://wikiwatcher.virgil.gr/pmcu}}
\end{description}

Die Methoden zur Datenextraktion und Visualisierung werden anschließend in eine Software integriert.
Die Gewinnung der von dieser Anwendung zu verarbeitenden Daten kann aus einer der folgenden Quellen erfolgen:

\begin{description}
\item[DB-Kopie] Monatlich angefertigte Moment-Aufnahmen der gesamten Wikipedia-Datenbank sind öffentlich verfügbar\footnote{\url{http://dumps.wikimedia.org}}. Eine solche Kopie enthält alle Artikel inklusive Versionsgeschichte und ist damit jedoch sehr groß\footnote{Eine Kopie der englischen Wikipedia-Datenbank umfasst derzeit 5,4 Terabyte.}.
\item[Artikelexport] Jeder einzelne oder mehrere Artikel der Wikipedia kann auch separat exportiert werden. Diese Daten umfassen ebenfalls die Versionsgeschichte und sind im Umfang bedeutend kleiner.
\item[Toolserver] Die Wikimedia Deutschland e.V. stellt Server bereit,\footnote{\url{http://toolserver.org}} welche einen direkten Zugang zu einer replizierten, schreibgeschützten Wikipedia-Datenbank ermöglichen. Die Nutzung eines solchen Servers vermeidet es zwar, eine eigene komplette Kopie der gesamten Wikipedia-Datenbank halten zu müssen, bedarf jedoch einer Anmeldung.
\end{description}

\subsection{Zugriff}

\todo{Tools and servers to access the articles.}


\section{Collective Authorship}

\todo{Introduce types of authors (roles) as well as methods to determine contribution/attribution}

\begin{todos}
    \item Autoren
    \item Bots
    \item Wer überlebt?
    \item Algorithmen, welche Unterschiede?
\end{todos}

\subsection{Relevant Edits}

\todo{Are all edits relevant? Edit wars? Bots?}



\section{Georeferences}

\begin{todos}
    \item registered vs. unregistered vs. bots vs. admins
    \item incorporate key findings of \cite{hardy2011volunteered} as laid out in chapter \ref{sec:georeference}
    \item IPs of unregistered users: Geo lookup
    \item Autoren-Profile: Information Extraction
    \item Geographische Zuordnung vom user profile
\end{todos}

Zur Bestimmung der Herkunft eines Autors bietet Wikipedia zwei direkte Ansätze: 
Für jeden Beitrag eines nicht registrierten Benutzers wird die IP-Adresse gespeichert, über die er Zugang zum Internet erlangt hat. 
Der zweite Ansatz betrifft die registrierten Benutzer.
Ihre IP-Adressen sind maskiert und nicht öffentlich zugänglich.\footnote{Eine kleine, von der Wikipedia-Community gewählte Nutzerschaft mit der Berechtigung \emph{checkuser} kann die Adressen demaskieren.}
Die registrierten Nutzer können jedoch auf ihrer \emph{user page} Informationen über ihre Person entweder als Freitext oder strukturiert in \emph{user boxes} veröffentlichen.
Letztere sind definierte Einheiten mit denen der Nutzer persönliche Eigenschaften wie Herkunftsland, gesprochene Sprachen oder wissenschaftliche Interessen kodifizieren kann.
Zusammen decken beide Ansätze jedoch nur einen Teil der Beiträge schreibenden Nutzerschaft ab.


\subsection{IP Look-up}

\begin{todos}
    \item Services
    \item Accuracy
    \item Active prevention by proxies and anonymizers: 
    \\ \fullcite{muir2006internet} 
    \\ \fullcite{muir2009internet}
    \\ \fullcite{duckham2005formal}
\end{todos}

Mit frei verfügbaren\footnote{Die vorgestellten Dienste haben ein tägliches Kontingent an Anfragen. Hilfstechniken wie Caching können diese Einschränkungen jedoch mindern.} Online-Diensten wie \term{Quova}\footnote{\url{http://developer.quova.com}} oder \term{geoplugin}\footnote{\url{http://www.geoplugin.com/webservices}} lässt sich für einen Großteil der IPs daraufhin das Herkunftsland bestimmen.

Im Bezug auf die Herkunft sind sowohl das Land als auch die Geo-Koordinaten interessant.  
Basierend auf der Versionsgeschichte würde für nicht registrierte Benutzer eine Gewinnung von Daten dann beispielsweise folgende Schritte durchlaufen:

\begin{quotation}
IP \RA Geolocation-Dienst \RA Koordinaten und Land
\end{quotation}


\subsection{Information Extraction}

\begin{todos}
    \item IE approach with Machine Learning \fullcite{xiao2004information}
    \item unsupervised IE: \fullcite{etzioni2005unsupervised}
    \item if city is mentioned, determine country (needs disambiguation, e.g. Berlin)
    \item coordinates are optional?
\end{todos}


\subsection{Geographic Profiling}

\begin{todos}
    \item \fullcite{lieberman2009you}
    \item \fullcite{hecht2010localness}
    \item from other fields such as criminal research: \\ \fullcite{snook2005complexity}
    \item feasibility, maybe just as enhancer
\end{todos}


\subsection{Consolidation}

\begin{todos}
    \item settle for a resolution
    \item some examples on accuracy for different countries
    \item clustering of origins: areas of influence
\end{todos}



\section{Visualization}

\begin{todos}
    \item Darstellung der geographischen Analyse
    \item per Wort, Satz, Artikel, Wort
\end{todos}

Auf Basis der strukturierten Daten in Form von Artikeln, Sätzen, Ländern, Koordinaten und Sprachen sollen nun Visualisierungen gefunden werden, welche die Fülle an Informationen zugänglich machen.
Mögliche Visualisierungen wären etwa:

\begin{labeling}{V2}
\item[V1] Revisionshistogramm à la Google Finance 
\item[V2] \emph{Heatmap} einer Landkarte mit Ursprüngen der Revisionen 
\item[V3] Netzwerkgrafik, die Metriken desselben Artikels in verschiedenen Sprachvarianten anzeigt
\item[V4] Dynamisches Blasendiagramm\footnote{\url{http://en.wikipedia.org/wiki/Motion_chart}} über die Entwicklung unterschiedlicher Sprachvarianten
\item[V5] \emph{Heatmap} des Artikels mit Stellen höchster Aktivität 
\item[V6] Landeskürzel für eine gegebene Textstelle
\item[V7] Edit wars on map, linking two or more places
\end{labeling}


\subsection{Goals}

\begin{todos}
    \item \fullcite{kjellin2010evaluating}
    \item Identify. Characteristics of an object.
    \item Locate. Absolute or relative position.
    \item Distinguish. Recognize as the same or different.
    \item Categorize. Classify according to some property (e.g., color, position, or shape).
    \item Cluster. Group same or related objects together.
    \item Distribution. Describe the overall pattern.
    \item Rank. Order objects of like types.
    \item Compare. Evaluate different objects with each other.
    \item Associate. Join in a relationship.
    \item Correlate. A direct connection.
\end{todos}


\subsection{Design}


\section{Data Model and System Overview}

\begin{todos}
    \item fetch article
    \item get revision history
    \item determine contributions
    \item transform to word attribution
    \item attach georeference 
\end{todos}


