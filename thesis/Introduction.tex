%************************************************
\chapter{Introduction}\label{ch:introduction}
%************************************************

\titlequote{If you are open to contributions from others, you generally end up with richer, better, more diverse and expert content than if you try to do it alone.\footnote{\fullcite{econ18904124}}}{Alan Rusbridger, editor of \company{The Guardian}}

\section{Context}

At the end of January 2011, when a wave of public protest spilled from Tunisia into Egypt, a small group of opposition parties and political activists called for a \glqq Day of Rage\grqq via Facebook, a social networking website.
By January 25th their Facebook group had more than 80,000 supporters who drew attention to and helped organize the country-wide protests that followed. 
As people rallied the streets day after day, the Egyptian government first limited access to Twitter, a micro-blogging service, before cutting Egypt off the internet completely on January 28th.\cite{econ18013760, szegypt}

In what came to be known as the Arab Spring, the use of online networks directly influenced the political development.
While Facebook played a part in organizing the protests, Twitter acted as an information channel during the demonstrations.
As the events unravelled, they were reflected by articles created on Wikipedia, an online encyclopedia.
Updated by the minute, the articles covering the protests formed a well of news reports.\cite{wikiegypt}
As ordinary people become producers of journalism the need arises to analyze these contributions. 
Specifically, this thesis focuses on the geographic origins of contributions to Wikipedia articles.

Wikipedia's free access and open editing policy as well as a quality level --- putting it ''head to head''\cite{giles2005internet} with Encyclopedia Britannica --- turned it into a hugely popular website\cite{wikipv}.
The server software used for the website, MediaWiki\footnote{\url{http://www.mediawiki.org}}, ensures that the effort to change an article is minimal.
Given an Internet connection and a web browser, anyone can add or edit an account of current events in a related article and publish it in a matter of seconds.

This form of news production turns the encyclopedia into a news channel that is constantly updated and corrected by an army of volunteers.
The result is a self-governed news source that lends itself the aura of authority and credibility of a knowledge reference.
At the same time a technophile public, that uses the Internet as an efficient means of news acquisition, can check facts on Wikipedia and act upon the consumed information.\cite[p. 424--427]{chadwick2009routledge}
Therefore the collective authorship of such a news medium could have a direct influence on the political decision process.

Political events are often limited to a country or region. 
This is reflected by the Wikipedia articles covering the Arab Spring: there is an overarching parent article\footnote{\url{http://en.wikipedia.org/wiki/2010-2011_Middle_East_and_North_Africa_protests}} as well as single articles covering the revolution in each of the affected countries, e.g. Egypt\footnote{\url{http://en.wikipedia.org/wiki/Egyptian_Revolution_of_2011}} and Libya\footnote{\url{http://en.wikipedia.org/wiki/2011_Libyan_uprising}}.
The latter also exemplifies how divided the political actors can be.
While nearly all revolutionaries welcomed the airstrikes, one faction was concerned foreign meddling and another one just opposed the deployment of ground troops.\cite{econ18290470}

The collective authorship could be equally divided\footnote{Despite Wikipedia's core policy to oblige everyone to write from a neutral point of view (\abbr{NPOV}), people regularly express opinions. The collision of opinions in a collectively written article can result in a prolonged series of an edit and its subsequent reversal by another person. The resulting edit pattern is known as an \emph{edit war}.\cite{suh2007us}} while at the same time creating a potential for further analysis.
Where do the first reports of an event originate?
As later iterations of edits turn these reports into historical accounts, Are later editors from the same country?


\todo{collaborative authorship will reflect this division, where do reports originate, iterations of edits turn it into an historical account, where do the editors post from}

Die kollektive Autorschaft eines Wikipedia-Artikels könnte ähnlich geteilt aussehen und würde damit erste Fragestellungen liefern, deren Analyse am Ende der Diplomarbeit ermöglicht werden soll: \graffito{Fragestellung} 
Kommen zum Beispiel die Verfasser eines Artikels über eine Revolution aus dem Land, das Schauplatz des Umbruches ist? 
Werden die Zustände vor Ort tatsächlich von \emph{innen} geschildert?
Lassen sich innerhalb eines Artikels Kontroversen und deren geographischer Ursprung identifizieren?
Ändert sich die Verteilung der Herkunft der Beiträge mit der Zeit? 
Wie verändert sich der Artikel nachdem ein Ereignis vorüber ist?

\todo{Why is the origin important? A place is linked to meaning, context etc.  }

Geschichte wird von Siegern geschrieben. 
Ob dieser Aphorismus ausgedient hat, wird die Diplomarbeit nicht beantworten können.
Ob die Bürgern eines Landes täglich oder sogar stündlich auf Wikipedia an ihrer Geschichte mitarbeiten, hingegen schon.


\section{Research Questions}

\todo{Focus on origin of contributions, how well it can be determined}

Im Rahmen dieser Diplomarbeit sollen Möglichkeiten untersucht werden, inwieweit der geographische Ursprung der Artikelbeiträge erfasst und aufbereitet werden kann, um etwa Dritte bei einer politischen Analyse eines Artikels zu unterstützen.
Eine Reihe von Visualisierungen soll dabei helfen, Aussagen über politische Zusammenhänge ableiten zu können, wie zum Beispiel die Identifikation der Einflussnehmerstaaten oder auch der Streitpunkte. 

\todo{can distribution of origins be related to place of event}

Die Nutzung dieser Software soll für einen gegebenen Artikel eine automatische, quantitative Auswertung durchführen und deren Ergebnisse geeignet darstellen, so dass zum Beispiel folgende Informationen erkennbar werden

\todo{can a statistical analysis be done to answer the main question}

\begin{labeling}{A2}
\item[A1] Ursprungsländer der Autoren und deren Anteil am Artikel
\item[A2] Zeitliche Entwicklung der Ursprünge der Autorschaft
\item[A3] Hauptstreitpunkte des Artikels
\item[A4] Vergleich der Sprachvarianten eines Artikels anhand einfacher Metriken wie Artikellänge, Anzahl der Autoren und Aktivitätslevel (Anzahl der Revisionen in einem festen Zeitintervall).
\end{labeling}

\todo{scope: only georeferencing, not behavior}



\section{Structure}

\todo{name the basic chapters and their function, one part = one paragraph} 

Das Kapitel \nameref{ch:foundation} beginnt mit einer Übersicht über bisherige Ergebnisse in den Gebieten \nameref{sec:contribution}, \nameref{sec:georeference} und \nameref{sec:visualization}.
Entlang dieser Überlegungen sollen bisherige Analysemethoden und Visualisierungen auf Eignung untersucht, gegebenenfalls weiterentwickelt und als Proof of Concept in einer Software umgesetzt werden.

\todo{thesis describes a method to help answer the research question}

Unter Einsatz der Software wird im Kapitel \nameref{ch:experiment} anhand einer Auswahl von Artikeln über politische Ereignisse eine solche Analyse durchgeführt werden, um die Kernfrage, ob ein Land seine Geschichte selbst schreibt, beispielhaft zu beantworten. 
Eine deskriptive, statistische Analyse einer Gruppe von politischen Artikeln schließt die Arbeit ab. 

\todo{results and conclusion}
