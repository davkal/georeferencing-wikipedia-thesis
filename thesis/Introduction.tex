%************************************************
\chapter{Introduction}\label{ch:introduction}
%************************************************

\section{Context}

At the end of January 2011, when a wave of public protest spilled from Tunisia into Egypt, a small group of opposition parties and political activists called for a \glqq Day of Rage\grqq via Facebook, a social networking website.
By January 25th their Facebook group had more than 80,000 supporters who drew attention to and helped organize the country-wide protests that followed. 
As people rallied the streets day after day, the Egyptian government first limited access to Twitter, a micro-blogging service, before cutting Egypt off the internet completely on January 28th.\cite{econ18013760, szegypt}

\todo{how do these events reflect on social media and wikipedia, see Econ. 'Back to the coffee house'}

The use of these information networks directly influenced the political development.
Facebook played a part in organizing the protests while Twitter acted as an information channel during the demonstrations.
As these events unravelled, they were reflected by articles created on Wikipedia, an online encyclopedia.
Updated by the minute, these articles formed a well of online reporting.\cite{wikiegypt}
This thesis will 
In der Diplomarbeit soll die Herkunft dieser Informationsbeiträge untersucht werden, um mithilfe der Ergebnisse eine Aussage über die Nutzung von Wikipedia als politisches Werkzeug machen zu können. 

\todo{why is wikipedia being used for this, what makes it special, why is it so easy to publish sth.}

Das freie Online-Lexikon, an dem jeder mitschreiben kann, zeichnet sich nicht nur durch eine hohe Qualität aus \cite{giles2005internet}, sondern erfreut sich auch an stetiger Popularität \cite{wikipv}. 
Dank der von Wikipedia eingesetzten Software MediaWiki\footnote{\url{http://www.mediawiki.org}} ist der Aufwand, an einem Artikel mitzuarbeiten, sehr gering.
Einen Internetzugang vorausgesetzt, kann jede Person die Entwicklungen von aktuellen Ereignissen im zugehörigen Artikel zeitnah beschreiben und innerhalb von Sekunden publizieren.

\todo{why should we care? wikipedia acts as a news channel while the aura of an encyclopedia lends it credibility and authority}

Diese Form der Mitarbeit erweitert das Nachschlagewerk zu einem Nachrichtenmedium, das ständig korrigiert und aktualisiert wird. 
Das Resultat ist eine einzigartige Quelle des Wissens, in dem sich jedoch die Möglichkeit einer Berichterstattung für jedermann mit der Autorität eines Lexikons vermischt. 
Eine technologieversierte Öffentlichkeit, die das Internet als effizientes Mittel zur Informationsgewinnung und -verbreitung ansieht, kann Wikipedia zum \emph{fact checking} nutzen und auf dieser Basis handeln.\cite[S. 424-427]{chadwick2009routledge}
Die Autorschaft eines solchen Mediums würde damit unmittelbaren Einfluss auf den politischen Entscheidungsprozess ausüben.

\todo{news are about events, events have a place, wikipedia articles cover events at a certain place}

Politische Ereignisse sind häufig auf ein Land oder eine Region begrenzt.
Dies spiegelt sich auch in den Artikeln über die Proteste in der arabischen Welt wider: es gibt sowohl einen zusammenfassenden \glqq Mutter-Artikel\grqq\footnote{\url{http://en.wikipedia.org/wiki/2010-2011_Middle_East_and_North_Africa_protests}} als auch einzelne Artikel über die Revolution in Ägypten\footnote{\url{http://en.wikipedia.org/wiki/Egyptian_Revolution_of_2011}} oder den Aufstand in Libyen\footnote{\url{http://en.wikipedia.org/wiki/2011_Libyan_uprising}}.

\todo{political events have different factions, express different views (despite NPOV), foreign meddling}

Am libyschen Beispiel ist auch erkennbar, dass solch ein politischer Umbruch ein äußerst empfindlicher Prozess ist.
Anfang März 2011 war die Gruppe der Aufständischen klar gespalten in Liberale und Islamisten.
Während beide Lager eine Flugverbotszone über Libyen forderten, war sich die Gemeinschaft über einen Einsatz von Bodentruppen uneinig.
Durch die Befürwortung eines Bodeneinsatzes liefen die Liberalen Gefahr, sowohl vom Regime als auch von den Islamisten als Handlanger ausländischer Mächte diskreditiert zu werden.\cite{econ18290470}

\todo{collaborative authorship will reflect this division, where do reports originate, iterations of edits turn it into an historical account, where do the editors post from}

Die kollektive Autorschaft eines Wikipedia-Artikels könnte ähnlich geteilt aussehen und würde damit erste Fragestellungen liefern, deren Analyse am Ende der Diplomarbeit ermöglicht werden soll: \graffito{Fragestellung} 
Kommen zum Beispiel die Verfasser eines Artikels über eine Revolution aus dem Land, das Schauplatz des Umbruches ist? 
Werden die Zustände vor Ort tatsächlich von \emph{innen} geschildert?
Lassen sich innerhalb eines Artikels Kontroversen und deren geographischer Ursprung identifizieren?
Ändert sich die Verteilung der Herkunft der Beiträge mit der Zeit? 
Wie verändert sich der Artikel nachdem ein Ereignis vorüber ist?

\todo{Why is the origin important? A place is linked to meaning, context etc.  }

Geschichte wird von Siegern geschrieben. 
Ob dieser Aphorismus ausgedient hat, wird die Diplomarbeit nicht beantworten können.
Ob die Bürgern eines Landes täglich oder sogar stündlich auf Wikipedia an ihrer Geschichte mitarbeiten, hingegen schon.


\section{Research Questions}

\todo{Focus on origin of contributions, how well it can be determined}

Im Rahmen dieser Diplomarbeit sollen Möglichkeiten untersucht werden, inwieweit der geographische Ursprung der Artikelbeiträge erfasst und aufbereitet werden kann, um etwa Dritte bei einer politischen Analyse eines Artikels zu unterstützen.
Eine Reihe von Visualisierungen soll dabei helfen, Aussagen über politische Zusammenhänge ableiten zu können, wie zum Beispiel die Identifikation der Einflussnehmerstaaten oder auch der Streitpunkte. 

\todo{can distribution of origins be related to place of event}

Die Nutzung dieser Software soll für einen gegebenen Artikel eine automatische, quantitative Auswertung durchführen und deren Ergebnisse geeignet darstellen, so dass zum Beispiel folgende Informationen erkennbar werden

\todo{can a statistical analysis be done to answer the main question}

\begin{labeling}{A2}
\item[A1] Ursprungsländer der Autoren und deren Anteil am Artikel
\item[A2] Zeitliche Entwicklung der Ursprünge der Autorschaft
\item[A3] Hauptstreitpunkte des Artikels
\item[A4] Vergleich der Sprachvarianten eines Artikels anhand einfacher Metriken wie Artikellänge, Anzahl der Autoren und Aktivitätslevel (Anzahl der Revisionen in einem festen Zeitintervall).
\end{labeling}

\todo{scope: only georeferencing, not behavior}



\section{Structure}

\todo{name the basic chapters and their function, one part = one paragraph} 

Das Kapitel \nameref{ch:foundation} beginnt mit einer Übersicht über bisherige Ergebnisse in den Gebieten \nameref{sec:contribution}, \nameref{sec:georeference} und \nameref{sec:visualization}.
Entlang dieser Überlegungen sollen bisherige Analysemethoden und Visualisierungen auf Eignung untersucht, gegebenenfalls weiterentwickelt und als Proof of Concept in einer Software umgesetzt werden.

\todo{thesis describes a method to help answer the research question}

Unter Einsatz der Software wird im Kapitel \nameref{ch:experiment} anhand einer Auswahl von Artikeln über politische Ereignisse eine solche Analyse durchgeführt werden, um die Kernfrage, ob ein Land seine Geschichte selbst schreibt, beispielhaft zu beantworten. 
Eine deskriptive, statistische Analyse einer Gruppe von politischen Artikeln schließt die Arbeit ab. 

\todo{results and conclusion}
