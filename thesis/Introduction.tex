%************************************************
\chapter{Einleitung}\label{ch:introduction}
%************************************************

\section{Motivation}

Als Ende Januar 2011 die Welle des öffentlichen Protestes von Tunesien nach Ägypten überschwappte, rief eine kleine Gruppe von Oppositionsparteien und politischer Aktivisten über die Website Facebook zu einem \glqq Tag des Zornes\grqq auf.
Am 25. Januar hatte die Facebook-Gruppe über 80.000 Unterstützer. 
In den landesweit organisierten Protesten gingen Zehntausende auf die Straße.
Aufgrund der andauernden Proteste schränkte die Regierung erst den Zugang zu sozialen Netzwerken wie Twitter ein, bevor sie am 28. Januar Ägypten vollständig vom Internet trennte.\cite{econ18013760, szegypt}
 
Die Nutzung dieser Informationsnetzwerke hatte direkten Einfluss auf die politischen Entwicklungen. 
Facebook diente zur Planung und Organisation der Proteste, wohingegen Twitter als Informationsmedium während der Proteste eingesetzt wurde. 
Parallel dazu wurden auf der Online-Enzyklopädie Wikipedia die Ereignisse minutiös festgehalten\cite{wikiegypt}, so dass diese Website als Sammelbecken für Informationen genutzt werden konnte. 
In der Diplomarbeit soll die Herkunft dieser Informationsbeiträge untersucht werden, um mithilfe der Ergebnisse eine Aussage über die Nutzung von Wikipedia als politisches Werkzeug machen zu können. 

Das freie Online-Lexikon, an dem jeder mitschreiben kann, zeichnet sich nicht nur durch eine hohe Qualität aus \cite{giles2005internet}, sondern erfreut sich auch an stetiger Popularität \cite{wikipv}. 
Dank der von Wikipedia eingesetzten Software MediaWiki\footnote{\url{http://www.mediawiki.org}} ist der Aufwand, an einem Artikel mitzuarbeiten, sehr gering.
Einen Internetzugang vorausgesetzt, kann jede Person die Entwicklungen von aktuellen Ereignissen im zugehörigen Artikel zeitnah beschreiben und innerhalb von Sekunden publizieren.

Diese Form der Mitarbeit erweitert das Nachschlagewerk zu einem Nachrichtenmedium, das ständig korrigiert und aktualisiert wird. 
Das Resultat ist eine einzigartige Quelle des Wissens, in dem sich jedoch die Möglichkeit einer Berichterstattung für jedermann mit der Autorität eines Lexikons vermischt. 
Eine technologieversierte Öffentlichkeit, die das Internet als effizientes Mittel zur Informationsgewinnung und -verbreitung ansieht, kann Wikipedia zum \emph{fact checking} nutzen und auf dieser Basis handeln.\cite[S. 424-427]{chadwick2009routledge}
Die Autorschaft eines solchen Mediums würde damit unmittelbaren Einfluss auf den politischen Entscheidungsprozess ausüben.

Politische Ereignisse sind häufig auf ein Land oder eine Region begrenzt.
Dies spiegelt sich auch in den Artikeln über die Proteste in der arabischen Welt wider: es gibt sowohl einen zusammenfassenden \glqq Mutter-Artikel\grqq\footnote{\url{http://en.wikipedia.org/wiki/2010-2011_Middle_East_and_North_Africa_protests}} als auch einzelne Artikel über die Revolution in Ägypten\footnote{\url{http://en.wikipedia.org/wiki/Egyptian_Revolution_of_2011}} oder den Aufstand in Libyen\footnote{\url{http://en.wikipedia.org/wiki/2011_Libyan_uprising}}.

Am libyschen Beispiel ist auch erkennbar, dass solch ein politischer Umbruch ein äußerst empfindlicher Prozess ist.
Anfang März 2011 war die Gruppe der Aufständischen klar gespalten in Liberale und Islamisten.
Während beide Lager eine Flugverbotszone über Libyen forderten, war sich die Gemeinschaft über einen Einsatz von Bodentruppen uneinig.
Durch die Befürwortung eines Bodeneinsatzes liefen die Liberalen Gefahr, sowohl vom Regime als auch von den Islamisten als Handlanger ausländischer Mächte diskreditiert zu werden.\cite{econ18290470}

Die kollektive Autorschaft eines Wikipedia-Artikels könnte ähnlich geteilt aussehen und würde damit erste Fragestellungen liefern, deren Analyse am Ende der Diplomarbeit ermöglicht werden soll: \graffito{Fragestellung} 
Kommen zum Beispiel die Verfasser eines Artikels über eine Revolution aus dem Land, das Schauplatz des Umbruches ist? 
Werden die Zustände vor Ort tatsächlich von \emph{innen} geschildert?
Lassen sich innerhalb eines Artikels Kontroversen und deren geographischer Ursprung identifizieren?
Ändert sich die Verteilung der Herkunft der Beiträge mit der Zeit? 
Wie verändert sich der Artikel nachdem ein Ereignis vorüber ist?

\todo{Why is the origin important? A place is linked to meaning, context etc.  }

Geschichte wird von Siegern geschrieben. 
Ob dieser Aphorismus ausgedient hat, wird die Diplomarbeit nicht beantworten können.
Ob die Bürgern eines Landes täglich oder sogar stündlich auf Wikipedia an ihrer Geschichte mitarbeiten, hingegen schon.


\section{Zielsetzung}

\begin{todos}
    \item Ansätze für Lösung des Problems
    \item Warum lösen diese das Problem?
\end{todos}

Während sich bisherige Studien eher auf das Kollaborationsverhalten und Qualitätsmetriken konzentrierten\footnote{Insbesondere ???}, steht in dieser Diplomarbeit der geographische Aspekt im Vordergrund.
In diesem Rahmen sollen Möglichkeiten untersucht werden, inwieweit der geographische Ursprung der Artikelbeiträge erfasst und aufbereitet werden kann, um etwa Dritte bei einer politischen Analyse eines Artikels zu unterstützen.
Eine Reihe von Visualisierungen soll dabei helfen, Aussagen über politische Zusammenhänge ableiten zu können, wie zum Beispiel die Identifikation der Einflussnehmerstaaten oder auch der Streitpunkte. 

Die Nutzung dieser Software soll für einen gegebenen Artikel eine automatische, quantitative Auswertung durchführen und deren Ergebnisse geeignet darstellen, so dass zum Beispiel folgende Informationen erkennbar werden:

\begin{labeling}{A2}
\item[A1] Ursprungsländer der Autoren und deren Anteil am Artikel
\item[A2] Zeitliche Entwicklung der Ursprünge der Autorschaft
\item[A3] Hauptstreitpunkte des Artikels
\item[A4] Vergleich der Sprachvarianten eines Artikels anhand einfacher Metriken wie Artikellänge, Anzahl der Autoren und Aktivitätslevel (Anzahl der Revisionen in einem festen Zeitintervall).
\end{labeling}



\section{Gliederung der Arbeit}

Das Kapitel \nameref{ch:foundation} beginnt mit einer Übersicht über bisherige Ergebnisse in den Gebieten \nameref{sec:contribution}, \nameref{sec:georeference} und \nameref{sec:visualization}.
Entlang dieser Überlegungen sollen bisherige Analysemethoden und Visualisierungen auf Eignung untersucht, gegebenenfalls weiterentwickelt und als Proof of Concept in einer Software umgesetzt werden.

Unter Einsatz der Software wird im Kapitel \nameref{ch:experiment} anhand einer Auswahl von Artikeln über politische Ereignisse eine solche Analyse durchgeführt werden, um die Kernfrage, ob ein Land seine Geschichte selbst schreibt, beispielhaft zu beantworten. 
Eine deskriptive, statistische Analyse einer Gruppe von politischen Artikeln schließt die Arbeit ab. 
