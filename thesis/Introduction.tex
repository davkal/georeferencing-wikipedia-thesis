%************************************************
\chapter{Introduction}\label{ch:introduction}
%************************************************

\titlequote{If you are open to contributions from others, you generally end up with richer, better, more diverse and expert content than if you try to do it alone.\footnote{\fullcite{econ18904124}}}{Alan Rusbridger, editor of \company{The Guardian}}

At the end of January 2011, when a wave of public protest spilled from Tunisia into Egypt, a small group of opposition parties and political activists called for a ``Day of Rage'' via Facebook, a social networking website.
By January 25th their Facebook group had more than 80,000 supporters who drew attention to and helped organize the country-wide protests that followed. 
As people rallied the streets day after day, the Egyptian government first limited access to Twitter, a micro-blogging service, before cutting Egypt off the internet completely on January 28th.\cite{econ18013760, szegypt}

In what came to be known as the Arab Spring, the use of online networks directly influenced the political development.
While Facebook played a part in organizing the protests, Twitter acted as an information channel during the demonstrations.
As the events unravelled, they were reflected by articles created on Wikipedia, an online encyclopedia.
Updated by the minute, the articles covering the protests formed a well of news reports.\cite{wikiegypt}
As ordinary people become producers of journalism the need arises to analyze these contributions. 
Specifically, this thesis focuses on the geographic origins of contributions to Wikipedia articles.

Wikipedia's free access and open editing policy as well as a quality level---putting it ``head to head''\cite{giles2005internet} with Encyclopedia Britannica---turned it into a hugely popular website\cite{wikipv}.
The server software used for the website, MediaWiki\footnote{\url{http://www.mediawiki.org}}, ensures that the effort to change an article is minimal.
Given an Internet connection and a web browser, anyone can add or edit an account of current events in a related article and publish it in a matter of seconds.

This form of news production turns the encyclopedia into a news channel that is constantly updated and corrected by an army of volunteers.
The result is a self-governed news source that lends itself the aura of authority and credibility of a knowledge reference.
At the same time a technophile public, that uses the Internet as an efficient means of news acquisition, can check facts on Wikipedia and act upon the consumed information.\cite[p. 424--427]{chadwick2009routledge}
Therefore the collective authorship of such a news medium could have a direct influence on the political decision process.

Political events are often limited to a country or region. 
This is reflected by the Wikipedia articles covering the Arab Spring: there is an overarching parent article\footnote{\url{http://en.wikipedia.org/wiki/2010-2011_Middle_East_and_North_Africa_protests}} as well as single articles covering the revolution in each of the affected countries, e.g. Egypt\footnote{\url{http://en.wikipedia.org/wiki/Egyptian_Revolution_of_2011}} and Libya\footnote{\url{http://en.wikipedia.org/wiki/2011_Libyan_uprising}}.
The latter also exemplifies how divided the political actors can be.
While nearly all revolutionaries welcomed the airstrikes, one faction was concerned foreign meddling and another one just opposed the deployment of ground troops.\cite{econ18290470}

The collective authorship could be equally divided\footnote{Despite Wikipedia's core policy to oblige everyone to write from a neutral point of view (\abbr{NPOV}), people regularly express opinions. The collision of opinions in a collectively written article can result in a prolonged series of an edit and its subsequent reversal by another person. The resulting edit pattern is known as an \emph{edit war}.\cite{suh2007us}} while at the same time creating a potential for further analysis.
Where do the first reports of an event originate?
As later iterations of revisions turn these reports into historical accounts, are these editors from the same country?
And more generally, to what extend is a collection of these articles written by volunteers located at the respective location of the event.

In this thesis I will propose a method to answer these questions.
By trying to determine the geographic origin of each edit to an article I will be able to calculate the geographic distribution of contributions.
This distribution will then be used to answer the questions above for either a single article or a collection.
\todo{Complete summary and key findings.}


\section{Structure}

\todo{complete over time, name the basic chapters and their function, one part = one paragraph} 

The next chapter \nameref{ch:foundation} provides the background about Wikipedia, article editing (\nameref{sec:contribution}), the application of geographic data (\nameref{sec:georeference}) and its presentation (\nameref{sec:visualization}).
The first part ends with \nameref{ch:hypotheses} where I propose the research questions that this thesis hopes to answer.

In \nameref{ch:apparatus} I will describe the tools available to be used in the method that will be applied to a host of articles in \nameref{ch:experiment}.
The findings will be presented in \nameref{ch:results}.
Followed by a discussion of their feasibility in \nameref{ch:conclusion}.