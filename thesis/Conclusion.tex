%************************************************
\chapter{Conclusion}\label{ch:conclusion}
%************************************************

In this thesis, I set out to determine whether articles of political events were written by the people most affected, i.e. in the event's proximity.
To investigate, I developed a web application that georeferenced contributions to those articles and analyzed them.
For the analysis, I presented a locating algorithm that can also locate a part of the registered users.

Then, I combined the georeferenced contributions with a source for text attribution. 
This allowed an article's content to be split into passages written by different located authors. 
In effect, this put a distance on a passage of text written by a located author.

The attribution of text to authors also allowed me to address the issue of text survival.
With this in mind, I devised two variants of a proximity metric, the signature distance, that are sensitive to edit survival.
The final variant, \term{t.surv}, calculated the signature distance of the text that survived to the current revision.

For three exemplary articles I showed how theses metrics can be used to analyze the localness of an article.
Then, I applied the same method on a set of articles about political conflicts.
Across the set, the results were mostly inconclusive.
Weak support was found for Wikipedia's role as a news medium while studying article creation.

But, as the experiments have shown, there is merit in an individual article analysis.
The difference in participation was starkest between the articles of the Bahraini uprising and the Libyan civil war.
Contributions from Libya to the article treating its current event, were almost absent.
On the other hand, tiny Bahrain contributed the bulk of the content to its article, truly writing its own history.


\section{Further research}

%\begin{todos}
%    \item Mobile contributions, smartphones
%    \item Privacy
%    \item Active prevention by proxies and anonymizers: 
%    \\ \fullcite{muir2006internet} 
%    \\ \fullcite{muir2009internet}
%    \\ \fullcite{duckham2005formal}
%    % user pages
%        \item IE approach with Machine Learning \fullcite{xiao2004information}
%    \item unsupervised IE: \fullcite{etzioni2005unsupervised}
%
%\end{todos}
Future research can build on the application and tweak it from the technology side, as suggested in \nameref{sec:enhancements}.
There is also the possibility to analyze the data already gathered in more refined ways.

The same hypotheses may be applicable to other types of articles than political ones.
The key requirements are that they have a location attribute and a time interval.
This is easily fulfilled by disaster articles, e.g. Fukushima Daiichi nuclear disaster\footurl{http://en.wikipedia.org/wiki/Fukushima_Daiichi_nuclear_disaster}{2011}{10}{31}.

Although \emph{location} is central to more abstract concepts like \term{Culture}\footurl{http://en.wikipedia.org/wiki/Culture}{2011}{10}{31} these subjects clearly defy being attributed with \emph{a} location.
Nevertheless, an analysis of the spatial distribution of contributors could be interesting. 

Finally, articles covering the same topic across various language editions could be analyzed to further investigate issues like language barrier---locals contributing only in their language---and information arbitrage as suggested by \textcite{adar2009information}.

