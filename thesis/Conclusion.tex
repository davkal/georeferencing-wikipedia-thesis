%************************************************
\chapter{Conclusion}\label{ch:conclusion}
%************************************************

\begin{todos}
	\item summarize method
	\item summarize results
	\item which hypotheses got confirmed? 
    \item wikipedia as news medium vs history book
\end{todos}


\section{Limitations}

\begin{todos}
    \item Mobile contributions, smartphones
    \item Privacy
    \item Active prevention by proxies and anonymizers: 
    \\ \fullcite{muir2006internet} 
    \\ \fullcite{muir2009internet}
    \\ \fullcite{duckham2005formal}
\end{todos}

\subsection{Political events}
The same hypotheses may be applicable to other types of articles than political ones.
The key requirements are that they have a location attribute and a time interval.
This is easily fulfilled by disaster articles, e.g. Fukushima Daiichi nuclear disaster\footurl{http://en.wikipedia.org/wiki/Fukushima_Daiichi_nuclear_disaster}{2011}{10}{31}.

\subsection{Article location}
Although \emph{location} is central to more abstract concepts like \term{Culture}\footurl{http://en.wikipedia.org/wiki/Culture}{2011}{10}{31} these subjects clearly defy being attributed with \emph{a} location.
Nevertheless, an analysis of the spatial distribution of contributors could be interesting. 

\subsection{Cross-language article growth}
The growth rates of articles covering the same topic across various language editions could be analyzed to further investigate issues like language barrier---locals contributing only in their language---and information arbitrage as suggested by \textcite{adar2009information}.

\section{Further Research}
