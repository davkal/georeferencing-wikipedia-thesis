%************************************************
\chapter{Experiments}\label{ch:experiment}
%************************************************

The application described in \nameref{ch:apparatus} will now be fed with articles. 
This chapter will demonstrate results for individual articles and give pointers to how they can be interpreted.
For a quantitative analysis testing the hypotheses on a set of articles, see the next chapter \nameref{ch:results}.

\section{Application run}

The experimental runs will be done on the following articles:

\begin{description}
  \item[Egypt-Rev] The article \term{2011 Egyptian revolution} details the protests leading to the overthrow of the government.
  According to the article, this event is still \term{ongoing}.
  \item[Orange-Rev] The \term{Orange Revolution} refers to the events surrounding the flawed Ukrainian presidential election in 2004. 
  This is an article that treats an event that has ended (November 2004 -- January 2005).
  \item[NYC] This article about the biggest city of the United States was included to represent articles that simply treat a place---with a lesser political dimension.
\end{description}

The general metrics of the three runs are shown in table \ref{examplearticles}.
The rows \emph{Located contributors} show the effectiveness of the georeferencing algorithm using different data sources.
The combined techniques of using the Poor Man's Checkuser database and the parsing of user pages increase the share of located contributors by 20\% to around 60\%.

It is also worth comparing the proximity metrics in the lower two groups of rows.
The values from the \emph{cumulative} group are based on the edit counts of all contributions.
For the second group of rows, only contributions were counted that provided text still \term{present in the latest revision}.
Regarding the distribution of unique The measures for the survived text show that contributions from some countries get edited away.
Far countries seem to be the 

\ctable[
    caption = Example articles,
    label   = examplearticles,
pos=h,doinside=\small ]{Xrrr}{
    \tnote{Contributors are all authors excluding bots.}
    \tnote[b]{Poor man's checkuser}
 }{\toprule         \tableheadline{} 
    	& \tableheadline{Egypt-Rev} 
    				& \tableheadline{Orange-Rev} 
							&  \tableheadline{NYC}\\ \midrule
    Size (\kb)			& 153.9 	&  42.6& 172.4 \\
    Created 		& 2011-01-25 	&  2004-11-25 & 2001-03-27 \\
    First in language 	& English 	&  Ukrainian & English \\
    Revisions 		& 7,039 	&  880 & 15,953 \\
    Contributors\tmark		& 1,539 	&  382 & 5,125 \\
     -- anonymous	& 748 	&  166 & 2,644 \\
    \midrule
    Located contributors (\%) 	& 58.3 & 59.2 & 61.4 \\
    -- anonymous (\%)		& 48.6 & 43.5 & 51.6 \\
    -- by \abbr{PMCU}\tmark[b]	(\%)		& 4.5 & 9.9 & 5.7 \\
    -- by user page (\%)	& 5.1 & 5.8 & 4.1 \\
    \midrule
    \emph{Cumulative} \\
    Unique countries of origin	& 72 & 38 & 83 \\
    Signature distance (km) 	& 4,731.5 & 6,090.4 & 2,922.5 \\
    \midrule
    \emph{Present in latest revision} \\
    Unique countries of origin 	& 36 & 24 & 23 \\
    Signature dist. e.surv (km) 	& 3,109.4 & 6,226.3 & 1,242.6 \\
    Signature dist. t.surv (km) 	& 367.9 & 3,825.9 & 721.8 \\
    Located text (\%) 	& 17.1 & 17.6 & 15.1 \\
    \bottomrule
}

\todo{
image after a complete run with markers
time it takes, 
requests made, 
small article, 
big article, 
images
}

\todo{table of runs with metrics}

\subsection{Georeferencing performance}

\todo{baseline: IP anonymous, PMCU, user page parsing}


\subsection{Distribution of origins vs survival}

\todo{
compare distribution of each group
pick an article from each group
}

\subsection{Proximity metrics}

\todo{
compare signature distance measures for each group
and sd, sd-esurv, sd-tsurf
}

\subsection{Motion chart}

\begin{todos}
   \item move up and down for each group
    \item clustering of origins: areas of influence
\end{todos}

\section{hypotheses analysis}

\todo{short table on the results for each, rows are hypothesis, columns are articles}
