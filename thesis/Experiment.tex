%************************************************
\chapter{Experiments}\label{ch:experiment}
%************************************************

\section{Data Set}

Using Wikipedia's category system. 
Articles are categorized by people.

\begin{todos}
    \item Choosing the ``right'' category 
    \item Can it be representative?
\end{todos}

Mithilfe der Export-Funktion von Artikeln lässt sich ein kleiner Datensatz generieren, an dem die Anwendung getestet werden kann.
Über dieselbe Export-Funktion kann auch eine Kategorie wie zum Beispiel \emph{Revolutions by country}\footnote{\url{http://en.wikipedia.org/wiki/Category:Revolutions_by_country}} angegeben werden. 
Als Ergebnis erhält man eine Sammlung von Artikeln über politische Ereignisse.


\section{Application}

\begin{todos}
    \item Beispielhafte Durchführung
    \item Sammlung der Ergebnisse
\end{todos}

Dabei könnte zum Beispiel sichtbar werden, dass sich ein bestimmter Artikel in verschiedenen Sprachvarianten unterschiedlich entwickelt. 
Falls ein Land mehrere offizielle Sprachen hat, könnte man diese entweder gruppiert oder einzeln im direkten Vergleich betrachten.
Ebenso könnten sich in Anlehnung an die \emph{edit wars} Streitpunkte anhand von Textstellen herauskristallisieren, die besonders umkämpft sind. 

