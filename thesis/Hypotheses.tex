%************************************************
\chapter{Hypotheses}\label{ch:hypotheses}
%************************************************

\todo{copy to introduction}
To gain insight on how Wikipedia is being used during and after political events\footnote{I will not try to argue what makes an event political but rather identify a set of events by picking a suitable category of articles.}, and ultimately whether articles covering the events are written by people that are most affected, I will propose a set of hypotheses aimed at different aspects of article production.

For this thesis I will use the event definition proposed by \textcite[243]{lewis1987philosophical}: 
\begin{quotation}
``An event is a localized matter of contingent fact. [\ldots] An event occurs in a particular spatiotemporal region.''
\end{quotation}
It follows that events must have clear spatial and temporal boundaries.
The spatial boundaries give it a location, distinguishing the where.
The temporal boundaries, namely the start and the end date, divide the event into three intervals: \emph{before}, \emph{during} and \emph{after}.\footnote{For ongoing events the end date will be the date of the analysis.}
Let me further denote the \emph{beginning} of an event as the first seven days of an event.\footnote{The interval was picked arbitrarily but acknowledges the fact that event dates in Wikipedia articles rarely carry a time attribute, therefor a shorter interval, say 24 hours, is less feasible.}
In conjunction with location this division into time intervals allows for a detailed look into \nameref{sec:articlecreation}, the level of \nameref{sec:participation} in following intervals as well as  \nameref{sec:textsurvival}.

\section{Article creation}\label{sec:articlecreation}

The use of Wikipedia concerning a political event starts with the creation of the article describing the event.
\todo{rewrite, not clear}
Due to the website's popularity I would expect the delay between the start date and the date of article creation to be rather short.
Moreover, considering the rising year-on-year Wikipedia usage in numbers of pages viewed\cite{wikipv}, this delay should be become shorter and shorter suggesting an increased use of Wikipedia as a news channel.
This leads to the following hypotheses:

\hypothesis{H1}{Articles are created with only a short delay after the start date of the event.}

\hypothesis{H2}{The more recent an article, the shorter is the delay between the event start and article creation.}

A user has the chance to create a new article in any of the 260-odd language editions of Wikipedia.
Although the English version is by far the biggest and most used, it would be interesting to see whether it is the prime choice to create the first article for a new event.
Shortly after the first article has been created, articles covering the same topic will be created across various editions of Wikipedia.
These articles are then being linked, mostly manually, via inter-wiki links.\footnote{\textcite{adar2009information} found that between two languages the inter-linking is not symmetrical, i.e. the number of out-links does not match the number of in-links. Links are either missing on one side or the respective topics are not congruous and the user intentionally left out one direction.}
When studying knowledge diversity across language editions, \textcite{hecht2010tower} found that the English edition is not the superset of concepts of all editions as was previously believed.
This means authors retain knowledge they consider important only to their compatriots. 
For a citizen of a country where English is not the first language creating an article becomes a political decision: should the author make the information available to fellow citizens or to a world-wide readership.
\todo{rewrite, not clear}
However, picking English, even though it is not an official language of the country where the event is happening, would further support Wikipedia's role as a news channel, thus:

\hypothesis{H3}{Articles are being created first in the English Wikipedia.}

Regarding the localness of contributions, \textcite[57]{hardy2011volunteered} has established that Wikipedians write about places in their proximity more often than distant ones.
His sample included only articles that have a geotag.
Naturally, articles about geographic places like towns and sights\footnote{According to \textcite{kittur2009s} articles about \term{geography and places} are third biggest group.} will dominate this sample.
Since my thesis is concerned with political events, I find this point to be worth revisiting\footnote{In addition, \textcite[61]{hardy2011volunteered} considered only anonymous users. Since creating an article is only allowed for registered users, his method has to be extended.} for the people most affected should be the ones creating the article, thus:

\hypothesis{H4}{Articles about political events are created by people in the events' proximity\footnote{\textcite[57]{hardy2011volunteered} defined proximity not in absolute terms, rather he considered the likeliness of authors being located less far than the average distance between an article and all its contributors.}.
}

Hypotheses 1--4 will only be tested against articles that were created as a reaction to an event that has already started. 
This excludes scheduled events like elections, e.g. \emph{Russian legislative election, 2011}\footurl{http://en.wikipedia.org/wiki/Russian_legislative_election,_2011}{2012}{1}{7} which was created 335 days before the election date, almost a year in advance.


\section{Participation}\label{sec:participation}

A Wikipedia article usually has more than one author.
Once it has been created, users from around the globe can edit an article collectively.
\textcite{viegas2004history} tried to find patterns in the revision history that would reveal certain aspects of collaboration or the lack thereof, e.g. discussions and vandalism.\footnote{Using their history flow visualization \textcite{viegas2004history} first identified patterns in single articles and later tried to statistically confirm their prevalence by analyzing the complete English corpus.} 
In respect to authorship, the researchers found the proportions of anonymous contributions differed strongly from page to page while showing no preference to any topic.
This inconclusive result and the age of the sample\footnote{\textcite{viegas2004history} used a dataset from May 2003.} merits further investigation. 

In 2007, \textcite{kittur2007power} found that a core of registered users is still doing the bulk of all edits.
However, anonymous users contribute considerable amounts of text. 
For accounts of political events, due to their dynamic nature, I expect a strong participation by unregistered users while the events are still unfolding:

\hypothesis{H5}{In the beginning of the event anonymous users contribute more than registered users.}

\hypothesis{H6}{For the duration of the event there are more local contributions than distant ones.}

When the political event is considered ``over'' its end date in the article changes from ``present'' to a calendar date.
In this unbounded and final phase I would expect the flood of contributions to subside and the content to consolidate when editors tighten prose or remove text they believe to be irrelevant.
Looking at the whole lifespan of an article I would also expect registered users to outnumber anonymous ones as suggested by \textcite{kittur2007power} and the spatial distribution of contributions to become less local.
Thus the final hypotheses:

\hypothesis{H7}{Articles of a political event that has ended will continuously shrink in size.}

\hypothesis{H8}{After an event has ended, there will be more contributions from registered users than from anonymous ones.}

\hypothesis{H9}{After an event has ended, the spatial distribution of the contributors will become less local.}

% only articles that have ended after article creation

\section{Text survival}\label{sec:textsurvival}

In \ref{sec:participation} the contributions are only treated in volume giving credit to each contributor.
However, when multiple authors write the same article, they do not only add text but also modify or even delete parts.
A user who reads an article will only see the text that has survived all edits after it was added.
\textcite{viegas2004history} found that early contributions have a high survival rate.
Recognizing this \term{first-mover advantage}, I suspect that accounts of political events show a strong localness in the beginning.
Thus the key hypotheses from \ref{sec:participation} have to be extended to reflect the spatial distributions of the contributions that make up the article:

\hypothesis{H10}{For the duration of the event the article text contains more local contributions than distant ones.}

\hypothesis{H11}{After an event has ended, the spatial distribution of the surviving contributions will become less local.}

This concludes the statement of the hypotheses. 
\todo{wording}
To test them, I will develop an \nameref{ch:apparatus} and the  \nameref{ch:experiment} in the next part.
