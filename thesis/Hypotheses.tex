%************************************************
\chapter{Hypotheses}\label{ch:hypotheses}
%************************************************

My effort to answer this thesis' main question, where the contributions to historic articles are coming from, will follow the path laid out by a set of questions:

\section{Research questions}

A central part of this thesis is the determination of the geographic origins of individual contributions.
In prior research regarding the spatial distribution of contributors, most notably \citetitle{hardy2011volunteered}, \textcite[see 3.4.2]{hardy2011volunteered} took only anonymous users into account.
Since anonymous contributions reveal the IP address their location can easily be approximated by using IP geolocation services.
He has firmly established that anonymous contributors are more likely to write about nearby places.
However, the above approach did not consider contributions done by registered users who, though being outnumbered by anonymous users, provide the majority of edits.
Based on alternative approaches like \textcite{lieberman2009you} and \textcite{engelhardt2010geographic}, I will investigate whether registered users can be included for a spatial analysis.
The search for the location of such a user will be satisfied when it yields a country which will be used as the origin of this user's contributions.
A key question is thus:

\question{Can a country of origin can be assigned to registered users?}

\textcite{hardy2011volunteered} also stopped short of how the spatial distribution changes over time.
The knowledge of time and location of the contributions suggest the following question:

\question{Does the spatial distribution of contributions change over time?}

Once the contributions have been georeferenced, the question of localness of the contributions can only be answered if the article itself can be geographically designated. 
A handful of studies exist, notably \cite{hardy2011volunteered,hecht2010localness}, that perform a spatial analysis on Wikipedia articles that are geo-tagged.
These articles contain geographic coordinates as part of the content, e.g. the article about the \term{Brandenburg Gate}\footurl{http://en.wikipedia.org/wiki/Brandenburg_Gate} in Berlin is tagged with the coordinates 52°30'58.58''N 13°22'39.80''E.
However, these articles are geographic in nature treating cities, rivers and places of interest. 
This thesis hopes to expand the spatial analysis of contributions to articles that have a location property only by association, e.g. the \term{Egyptian Revolution of 2001}\footurl{http://en.wikipedia.org/wiki/Egyptian_Revolution_of_2011}, that clearly happened in \term{Egypt}\footurl{http://en.wikipedia.org/wiki/Egypt}.
To allow for an analysis of historic events it is necessary to investigate the articles in a similar manner as the registered users:

\question{Can a location be assigned to an historic article?}

Finally, an article with a location can be related to the spatial distribution of its contributors.
In the context of historic articles this suggests the final two questions:

\question{Are historic articles written by their respective citizens?}

\section{Scope}

\subsection{Historic articles}
The same questions may be applicable to other types of articles than historic ones.
However, the latter satisfy the requirement of having a location attribute and were therefor referred to throughout this thesis.

\subsection{Article location}
Although \emph{location} is central to more abstract concepts like \term{Culture}\footurl{http://en.wikipedia.org/wiki/Culture} these subjects clearly defy being attributed with \emph{a} location.
Nevertheless, an analysis of the spatial distribution of contributors could be interesting. 
