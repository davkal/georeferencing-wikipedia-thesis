%************************************************
\chapter{Hypotheses}\label{ch:hypotheses}
%************************************************

To gain insight on how Wikipedia is being used during and after political events, and ultimately whether articles covering the events are written by people that are most affected, I will propose a set of hypotheses aimed at different aspects of article production.

For this thesis I will define an event as an occurrence that has an effect on the public affairs of a country.
Let then an occurrence be marked by a location, a start date and, if not ongoing, an end date. 
Further, let the beginning of an event be defined as the first quarter of time between its start and its end\footnote{For ongoing events the end date will be the present.}.
In conjunction with the spatial distribution this division into time intervals allows for a detailed look into \nameref{sec:articlecreation}, the level of \nameref{sec:participation} in following intervals as well as  \nameref{sec:textsurvival}.

\section{Article creation}\label{sec:articlecreation}

The use of Wikipedia concerning a political event starts with the creation of the article describing the event.
Due to the website's popularity I would expect the delay between the start date and the date of article creation to be rather short.
Moreover, considering the rising year-on-year Wikipedia usage\cite{wikipv}, this delay should be become shorter and shorter suggesting an increased use of Wikipedia as a news channel.
This leads to the following hypotheses:

\hypothesis{Articles are created with only a short delay after the start date of the event.}

\hypothesis{The more recent an article, the shorter is the delay between the event start and article creation.}

A user has the chance to create a new article in any of the 260-odd language editions of Wikipedia.
Although the English version is by far the biggest and most used, it would be interesting to see whether it is the prime choice to create the first article for a new event.
Shortly after the first article has been created, articles covering the same topic will be created across various editions of Wikipedia.
These articles are then being linked, mostly manually, via inter-wiki links.\footnote{\textcite{adar2009information} found that between two languages the inter-linking is not symmetrical, i.e. the number of out-links does not match the number of in-links. Links are either missing on one side or the respective topics are not congruous and the user intentionally left out one direction.}
When studying knowledge diversity across language editions, \textcite{hecht2010tower} found that the English edition is not the superset of concepts of all editions as was previously believed.
However, picking English, even though it is not an official language of the country where the event is happening, would further support Wikipedia's role as a news channel, thus:

\hypothesis{Articles are being created first in the English Wikipedia.}

Regarding the localness of contributions, \textcite[57]{hardy2011volunteered} has established that Wikipedians write about places in their proximity more often than distant ones.
His sample included only articles that have a geotag.
Naturally, articles about geographic places like towns and sights\footnote{According to \textcite{kittur2009s} articles about \term{geography and places} are third biggest group.} will dominate this sample.
Since my thesis is concerned with political events, I find this point to be worth revisiting\footnote{In addition, \textcite[61]{hardy2011volunteered} considered only anonymous users. Since creating an article is only allowed for registered users, his method has to be extended.} for the people most affected should be the ones creating the article, thus:

\hypothesis{Articles about political events are created by people in the events' proximity\footnote{\textcite[57]{hardy2011volunteered} defined proximity not in absolute terms, rather he considered the likeliness of authors being located less far than the average distance between an article and all its contributors.}.
}


\section{Participation}\label{sec:participation}

A Wikipedia article usually has more than one author.
Once it has been created, users from around the globe can edit an article collectively.
\textcite{viegas2004history} tried to find patterns in the revision history that would reveal certain aspects of collaboration or the lack thereof, e.g. discussions and vandalism.\footnote{Using their history flow visualization \textcite{viegas2004history} first identified patterns in single articles and later tried to statistically confirm their prevalence by analyzing the complete English corpus.} 
In respect to authorship, the researchers found the proportions of anonymous contributions differed strongly from page to page while showing no preference to any topic.
This inconclusive result and the age of the sample\footnote{\textcite{viegas2004history} used a dataset from May 2003.} merits further investigation. 

In 2007, \textcite{kittur2007power} found that a core of registered users is still doing the bulk of all edits.
However, anonymous users contribute considerable amounts of text. 
For accounts of political events, due to their dynamic nature, I expect a strong participation by unregistered users while the events are still unfolding:

\hypothesis{In the beginning of the event anonymous users contribute more than registered users.}

\hypothesis{For the duration of the event there are more local contributions than distant ones.}

When the political event is considered ``over'' its end date in the article changes from ``present'' to a calendar date.
In this unbounded and final phase I would expect the flood of contributions to subside and the content to consolidate when editors tighten prose or remove text they believe to be irrelevant.
Looking at the whole lifespan of an article I would also expect registered users to outnumber anonymous ones as suggested by \textcite{kittur2007power} and the spatial distribution of contributions to become less local.
Thus the final hypotheses:

\hypothesis{Articles of a political event that has ended will continuously shrink in size.}

\hypothesis{After an event has ended, there will be more contributions from registered users than from anonymous ones.}

\hypothesis{After an event has ended, the spatial distribution of the contributors will become less local.}

\section{Text survival}\label{sec:textsurvival}

In \ref{sec:participation} the contributions are only treated in volume giving credit to each contributor.
However, when multiple authors write the same article, they do not only add text but also modify or even delete parts.
A user who reads an article will only see the text that has survived all edits after it was added.
\textcite{viegas2004history} found that early contributions have a high survival rate.
Recognizing this \term{first-mover advantage}, I suspect that accounts of political events show a strong localness in the beginning.
Thus the key hypotheses from \ref{sec:participation} have to be extended to reflect the spatial distributions of the contributions that make up the article:

\hypothesis{For the duration of the event the article text contains more local contributions than distant ones.}

\hypothesis{After an event has ended, the spatial distribution of the surviving contributions will become less local.}

\section{Scope and limitations}

\subsection{Political events}
The same hypotheses may be applicable to other types of articles than political ones.
The key requirements are that they have a location attribute and a time interval.
This is easily fulfilled by disaster articles, e.g. Fukushima Daiichi nuclear disaster\footurl{http://en.wikipedia.org/wiki/Fukushima_Daiichi_nuclear_disaster}.

\subsection{Article location}
Although \emph{location} is central to more abstract concepts like \term{Culture}\footurl{http://en.wikipedia.org/wiki/Culture} these subjects clearly defy being attributed with \emph{a} location.
Nevertheless, an analysis of the spatial distribution of contributors could be interesting. 

\subsection{Cross-language article growth}
The growth rates of articles covering the same topic across various language editions could be analyzed to further investigate issues like language barrier---locals contributing only in their language---and information arbitrage as suggested by \textcite{adar2009information}.