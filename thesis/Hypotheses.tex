%************************************************
\chapter{Hypotheses}\label{ch:hypotheses}
%************************************************

To gain insight on how Wikipedia is being used during and after political events, and ultimately whether contributions to their respective articles are written by people that are most affected, I will propose a set of hypotheses aimed at different aspects of article production.

For this thesis I will define an event as an occurrence that has an effect on the public affairs of a country.
Let then an occurrence be marked by a location, a start date and, if not ongoing, an end date. 
Further, let the beginning of an event be defined as the first quarter of time between its start and its end\footnote{For ongoing events the end date will be the present.}.
In conjunction with the spatial distribution this division into time intervals allows for a detailed look into \nameref{sec:articlecreation}, the level of  \nameref{sec:participation} in following interval as well as  \nameref{sec:textsurvival}.

\section{Article creation}\label{sec:articlecreation}

The use of Wikipedia concerning a political event starts with the creation of the article describing the event.
In general one would expect the delay between the start date and the date of article creation to be rather short and, considering a rising year-on-year usage\cite{wikipv}, to shorten with time.
This leads to the following hypotheses:

\hypothesis{Articles are created with only a short delay after the beginning of the event.}

\hypothesis{The delay between the beginning of the event and its article creation decreases over time.}

\textcite[57]{hardy2011volunteered} has established that Wikipedians write about places in their proximity more often than distant ones.
His sample included all articles that have a geotag.
Naturally, articles about geographic places like towns and sights\footnote{According to \textcite{kittur2009s} these articles are third biggest group.} will dominate this sample.
Since this thesis is concerned with political events, this point is worth revisiting\footnote{In addition, \textcite[61]{hardy2011volunteered} considered only anonymous users. Since creating an article is only allowed for registered users, his method has to be extended.} for the people most affected should be the ones creating the article, thus:

\hypothesis{Articles about political events are created by people in the events' proximity.}


\section{Participation}\label{sec:participation}

A Wikipedia article usually has more than one author.
Once it has been created, users from around the globe can edit an article collectively.
\textcite{viegas2004history} tried to find patterns in the revision history that would reveal certain aspects of collaboration or the lack thereof, e.g. discussions and vandalism.\footnote{Using their history flow visualization \textcite{viegas2004history} first identified patterns in single articles and later tried to statistically confirm their prevalence by analyzing the complete English corpus.} 
In respect to authorship, the researchers found the proportions of anonymous contributions differed strongly from page to page while showing no preference to any topic.
This inconclusive result and the age of the sample\footnote{\textcite{viegas2004history} used a dataset from May 2003.} merits further investigation. 

Overall registered users have the biggest number of edits which could be due to their revising nature.\cn
However, anonymous users contribute considerable amounts of text. 
For accounts of political events, due to their dynamic nature, I expect a strong participation by unregistered users while the events are still unfolding.

\hypothesis{In the beginning of the event anonymous users contribute more than registered users.}

\hypothesis{For the duration of the event there are more local contributions than distant ones.}

When the political is considered ``over'' its end date in the article changes from ``present'' to a calendar date.
In this final phase, I would expect the contributions to subside and the content to consolidate when editors tighten prose or remove text they believe to be irrelevant.
Given their more revisory nature\cn, I would expect registered users to outnumber anonymous ones and the spatial distribution contributions to become less local.
Thus the final hypotheses:

\hypothesis{Articles of a political event that has ended will continuously shrink in size.}

\hypothesis{After an event has ended, there will be more contributions from registered users than from anonymous ones.}

\hypothesis{After an event has ended, the spatial distribution of the contributors will become less local.}

\section{Text survival}\label{sec:textsurvival}

In \nameref{sec:participation} the contributions are only treated in volume giving credit to each contributor.
However, when multiple authors write the same article, they do not only add text but also modify or even delete parts.
A user who reads an article will only see the text that has survived all edits after it was added.
\textcite{viegas2004history} found that early contributions have a high survival rate.
Recognizing this \term{first-mover advantage}, I suspect that accounts of political events show a strong localness in the beginning.
Thus the key hypotheses from \nameref{sec:participation} have to be extended to reflect the spatial distributions of the contributions that make up the article:

\hypothesis{For the duration of the event the article text contains more local contributions than distant ones.}

\hypothesis{After an event has ended, the spatial distribution of the surviving contributions will become less local.}

\section{Scope}

\subsection{Political events}
The same hypotheses may be applicable to other types of articles than political ones.
However, the latter satisfy the requirement of having a location attribute and a time interval.

\subsection{Article location}
Although \emph{location} is central to more abstract concepts like \term{Culture}\footurl{http://en.wikipedia.org/wiki/Culture} these subjects clearly defy being attributed with \emph{a} location.
Nevertheless, an analysis of the spatial distribution of contributors could be interesting. 