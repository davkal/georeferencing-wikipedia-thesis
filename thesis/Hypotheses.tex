%************************************************
\chapter{Hypotheses}\label{ch:hypotheses}
%************************************************

\section{Research questions}

A handful of studies exist that analyze Wikipedia's geographical coverage based on articles that are geo-tagged.
These articles contain geographic coordinates as part of the article content, e.g. the article about the Brandenburg Gate\footurl{http://en.wikipedia.org/wiki/Brandenburg_Gate} in Berlin contains its coordinates.
These articles are mostly about geographic manifestations like cities, rivers and other places of interest but not about much else. 
\todo{cite paper about coverage}
A central part of this thesis is the determination of the geographic origins of individual contributions.
A key research question for all following is thus:

\question{What is the average ratio of contributions for which the geographic origins can be determined vs. all contributions to an article?}

Once the contributions have been georeferenced, the question of locality of the contributions can only be answered if the article itself can be geographically designated. 
This is easy for the aforementioned geo-tagged articles concerning places, but will be more difficult for articles about historical events but according to \textcite{buscaldi2007comparison} not impossible.
For some abstract concepts like \emph{culture}\footurl{http://en.wikipedia.org/wiki/Culture} it will even be futile.
To allow for an analysis of historic events it is necessary to investigate the articles in a similar manner as the contributions:

\question{What is the ratio of articles for which the geographic origins can be determined vs. all articles?}

Even if an article does not have a location property, an analysis of the origins of its contributions can be desirable.
According to \textcite{anthony2005explaining} the majority of high quality content is produced either by one-time users, the long tail of contributors, or registered power users who act as vigilantes for their favorite articles. 
Given the strong bias shown by \textcite{ortega2009wikipedia} 
Using the georeferenced contributions one can analyze how the spatial distribution of their origins changes over time.
In fact, it seems unlikely that 

\question{How much does the distribution of origins change over time?}



\todo{can a statistical analysis be done to answer the main question}

\question{Do people write their own history?}

 
\section{Scope}

\subsection{Article location}

\subsection{Behavior}
