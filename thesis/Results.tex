%************************************************
\chapter{Results}\label{ch:results}
%************************************************

This chapter present the quantitative analysis of a set of articles of political events from the English Wikipedia.
First, I will describe the \nameref{sec:datasets} and the difficulties in picking the ``right'' category of articles.
\todo{whats happening here}

\section{Data set}\label{sec:datasets}

Unfortunately, \term{political events} are a rather vague concept judging by the contents of the Wikipedia category of the same name\footurl{http://en.wikipedia.org/wiki/Category:Political_events}{2012}{02}{01}.
It includes only one page at the top-level, \term{No Berlusconi day}.
It contains numerous subcategories, however, ranging from \term{Political riots} and \term{Protests} to \term{Political party assemblies}.

In contrast to the categories system, templates offer a way to identify more homogenous groups of articles.\footnote{See also \nameref{sub:templates}.}
Articles that embed a template, e.g. \term{Infobox civil conflict}, can be identified and grouped, resulting in a set of articles that are very likely to treat civil conflicts. 

In order to find a suitable set of articles, I will combine both approaches.
From the English Wikipedia I will handpick categories and templates and combine their articles into one data set.

Since this process involves a manual selection, articles are not selected at random.
As a result, statistics derived from the analysis cannot be representative of all articles treating political events---let alone the complete corpus of the English Wikipedia.
On the bright side, the handpicked categories contain articles that were themselves manually chosen by the community to be included those categories.

\subsection{Selected categories and templates}

Wikipedia's category system suffers from several forms of overcategorization.\footnote{For an overview of issues see \weburl{http://en.wikipedia.org/wiki/Wikipedia:OCAT}{2012}{02}{04}} 
Some category sub-trees are too fine-grained and have numerous  levels with only a small number articles.
When moving along the tree in search for articles of political conflicts, the potential to go astray is rather high, e.g. the category \term{Protest}\footurl{http://en.wikipedia.org/wiki/Category:Protests}{2012}{02}{04} has the subcategory \term{Protest songs}.
There is also considerable overlap between sub-trees, e.g. \term{Protests by country} and \term{Protests by year}.
As a compromise, the category traversal will only yield the pages of a given category and all pages of its subcategories.

\term{Protests} and its subcategories are exhaustive and produce a handful of candidates.
To these I will add the category \term{Arab Spring} and the template \term{Infobox civil conflict}\footnote{This template is also the preferred infobox template of the Occupy movement. \weburl{http://en.wikipedia.org/wiki/Occupy_movement}{2012}{02}{04}} to assert that the articles detailed in \nameref{ch:experiment} are part of the data set.
The final data set \entity{conflictsraw} combines articles from the following categories and templates:

\begin{itemize}
  \item Protests by year % good, 45.5 % of 143
  \item Protests by country
  \item 2000s riots by year
  \item 2010s riots by year
  \item Arab Spring
  \item Infobox civil conflict (template)
\end{itemize}

Other notable categories that I investigated but not included, are 
\term{21st-century conflicts}\footnote{Only 10\% of the 833 articles would have qualified. \weburl{http://en.wikipedia.org/wiki/Category:21st-century_conflicts}{2012}{02}{04}}, 
\term{Revolutions categories}\footnote{This category and its subcategories contain too many articles about people. \weburl{http://en.wikipedia.org/wiki/Category:Revolutions_by_country}{2012}{02}{04}}, as well as the templates 
\term{Infobox military conflict}\footnote{A huge set of 10290 articles that is dominated by historic battles. Less than 6\% of these articles would have qualified. \weburl{http://en.wikipedia.org/wiki/Template:Infobox_military_conflict}{2012}{02}{04}} and 
\term{Infobox election}\footnote{These articles are usually created long before the election date, as elections tend to be scheduled. \weburl{http://en.wikipedia.org/wiki/Template:Infobox_election}{2012}{02}{04}}.


\subsection{Data set preparation}

When articles for \entity{conflictsraw} are retrieved, they are only added if they are not already in the set. 
Therefore, all articles in the set are unique.
The titles of all articles in \entity{conflictsraw} are listed in the appendix, see \nameref{sec:conflictsraw}.

Then, the all articles in \entity{conflictsraw} are tested whether they fulfill the \nameref{sub:articlerequirements} for an analysis.
Of the 742 articles in \entity{conflictsraw} XXX passed this first test.
The main reason for rejection was an event date before 2002. 
For the distribution of failed requirements \imgref{group-requirements}.
All articles that fulfill the requirements are included in the data set \entity{conflicts}, see \nameref{sec:conflicts} in the appendix for a list of titles.

\img{group-requirements}{Distribution of failed article requirements}


%\subsection{Political article vs. place article}
%
%\todo{
%To compare the date with spatial articles preferred in previous studies, a reference group is introduced
%template info box settlement is used by towns all over the world
%population is big 261,000+ articles, pick a random sample
%compare locating performance, sd metrics.
%}

\section{Characteristics of the set}

Each article in the set has a location attribute. 
Therefore the spatial distribution of these articles can be plotted on a map.

\imgwide{group-map}{Spatial distribution of articles in data set \entity{conflicts}.}

\todo{timeline}

During the data set analysis I measured the similar metrics as in the \nameref{ch:experiment}.
The results are in table \ref{groupmetrics}.

\ctable[
    caption = Metrics for articles in \entity{conflicts},
    	label   = groupmetrics,
pos=h,doinside=\small ]{Xlll}{
%    \tnote{Only applicable to a group of articles.}
%    \tnote[b]{Unable to locate creator.}
%    \tnote[c]{Share of located contributions.}
%    \tnote[d]{No correlation found.}
%    \tnote[e]{A negative date is the result of an article being merged with another article treating earlier events.}
 }{\toprule 
%    						& \tableheadline{Egypt-Rev} 
%    								& \tableheadline{Libya-war} 
%											& \tableheadline{Bahrain-up}\\ \midrule
    Age 				&  \\
    Size (\kb)			&  \\
    Revisions 			&  \\
    Contributors\tmark		&  \\
     -- anonymous		&  \\
    \midrule
    Located contributors (\%) &  \\
    -- anonymous (\%)		&  \\
    -- by \abbr{PMCU}	(\%) &  \\
    -- by user page (\%)	&  \\
    \midrule
    \emph{Cumulative} \\
    Unique countries of origin	&  \\
    Signature distance (km) 	&  \\
    \midrule
    \emph{Present in latest revision} \\
    Unique countries of origin 	&  \\
    Signature dist. e.surv (km) 	& \\
    Signature dist. t.surv (km) 	& \\
     e.surv index 				&  \\
     t.surv index		&  \\
    Located text (\%) 	&  \\
    \bottomrule
}

\todo{Compared to the example articles \abbr{egypt-rev}}


For each of the hypotheses the articles in the set are tested whether they qualify, i.e. have enough data for an analysis, \imgref{group-qualifications}. 
The requirements for each hypothesis are defined in \nameref{hypothesesanalysis}.
The test for \hyp{3} to the requirement is the most restrictive as it rules out events from countries with English as an official language.

\img{group-qualifications}{Article qualification for each hypothesis}

\section{H1 -- H4: Article creation}

\img{group-h1}{Distribution of creation delays in days}
\img{group-h2}{Creation delays over time}
\img{group-h3}{Language editions chosen for first article}
\img{group-h4}{Creator localness}

% H1 susceptible to outliers


\section{H5 -- H8: Participation}

\boximg{group-h5}{ Anon. }{group-h6}{Locals}{group-h56}{Distribution of contributor ratios in early revisions}
\boximg{group-h7}{Anon.}{group-h8}{Local}{group-h78}{Distribution of correlation coefficients for the share of anonymous and local authors, respectively}


\section{H9/H10 Survival}

\boximg{group-h9}{$\overline{I_{\text{e.surv}}}$}{group-h10}{$\overline{I_{\text{t.surv}}}$}{group-h910}{Distribution of means of localness indexes}


% H10 when locals low in the beginning but high text survival, its not first mover advantage, but better text quality.
