%************************************************
\chapter{Results}\label{ch:results}
%************************************************

This chapter present the quantitative analysis of a set of articles of political events from the English Wikipedia.
First, I will describe the \nameref{sec:datasets} and the difficulties in picking the ``right'' category of articles.
\todo{whats happening here}

\section{Data set}\label{sec:datasets}

Unfortunately, \term{political events} are a rather vague concept judging by the contents of the Wikipedia category of the same name\footurl{http://en.wikipedia.org/wiki/Category:Political_events}{2012}{02}{01}.
It includes only one page at the top-level, \term{No Berlusconi day}.
It contains numerous subcategories, however, ranging from \term{Political riots} and \term{Protests} to \term{Political party assemblies}.

In contrast to the categories system, templates offer a way to identify more homogenous groups of articles.\footnote{See also \nameref{sub:templates}.}
Articles that embed a template, e.g. \term{Infobox civil conflict}, can be identified and grouped, resulting in a set of articles that are very likely to treat civil conflicts. 

In order to find a suitable set of articles, I will combine both approaches.
From the English Wikipedia I will handpick categories and templates and combine their articles into one data set.

Since this process involves a manual selection, articles are not selected at random.
As a result, statistics derived from the analysis cannot be representative of all articles treating political events---let alone the complete corpus of the English Wikipedia.
On the bright side, the handpicked categories contain articles that were themselves manually chosen by the community to be included those categories.

\subsection{Selected categories and templates}

Wikipedia's category system suffers from several forms of overcategorization.\footnote{For an overview of issues see \weburl{http://en.wikipedia.org/wiki/Wikipedia:OCAT}{2012}{02}{04}} 
Some category sub-trees are too fine-grained and have numerous  levels with only a small number articles.
When moving along the tree in search for articles of political conflicts, the potential to go astray is rather high, e.g. the category \term{Protest}\footurl{http://en.wikipedia.org/wiki/Category:Protests}{2012}{02}{04} has the subcategory \term{Protest songs}.
There is also considerable overlap between sub-trees, e.g. \term{Protests by country} and \term{Protests by year}.
As a compromise, the category traversal will only yield the pages of a given category and all pages of its subcategories.

\term{Protests} and its subcategories are exhaustive and produce a handful of candidates.
To these I will add the category \term{Arab Spring} and the template \term{Infobox civil conflict}\footnote{This template is also the preferred infobox template of the Occupy movement. \weburl{http://en.wikipedia.org/wiki/Occupy_movement}{2012}{02}{04}} to assert that the articles detailed in \nameref{ch:experiment} are part of the data set.
The final data set \entity{conflictsraw} combines articles from the following categories and templates:

\begin{itemize}
  \item Protests by year % good, 45.5 % of 143
  \item Protests by country
  \item 2000s riots by year
  \item 2010s riots by year
  \item Arab Spring
  \item Infobox civil conflict (template)
\end{itemize}

Other notable categories that I investigated but not included, are 
\term{21st-century conflicts}\footnote{Only 10\% of the 833 articles would have qualified. \weburl{http://en.wikipedia.org/wiki/Category:21st-century_conflicts}{2012}{02}{04}}, 
\term{Revolutions categories}\footnote{This category and its subcategories contain too many articles about people. \weburl{http://en.wikipedia.org/wiki/Category:Revolutions_by_country}{2012}{02}{04}}, as well as the templates 
\term{Infobox military conflict}\footnote{A huge set of 10290 articles that is dominated by historic battles. Less than 6\% of these articles would have qualified. \weburl{http://en.wikipedia.org/wiki/Template:Infobox_military_conflict}{2012}{02}{04}} and 
\term{Infobox election}\footnote{These articles are usually created long before the election date, as elections tend to be scheduled. \weburl{http://en.wikipedia.org/wiki/Template:Infobox_election}{2012}{02}{04}}.


\subsection{Data set preparation}

When articles for \entity{conflictsraw} are retrieved, they are only added if they are not already in the set. 
Therefore, all articles in the set are unique.
The titles of all articles in \entity{conflictsraw} are listed in the appendix, see \nameref{sec:conflictsraw}.

Then, the all articles in \entity{conflictsraw} are tested whether they fulfill the \nameref{sub:articlerequirements} for an analysis.
Of the 742 articles in \entity{conflictsraw} XXX passed this first test.
The main reason for rejection was an event date before 2002. 
For the distribution of failed requirements \imgref{group-requirements}.
All articles that fulfill the requirements are included in the data set \entity{conflicts}, see \nameref{sec:conflicts} in the appendix for a list of titles.

\img{group-requirements}{Distribution of failed article requirements}


%\subsection{Political article vs. place article}
%
%\todo{
%To compare the date with spatial articles preferred in previous studies, a reference group is introduced
%template info box settlement is used by towns all over the world
%population is big 261,000+ articles, pick a random sample
%compare locating performance, sd metrics.
%}

\section{Characteristics of the set}

The analysis of the set yielded some general metrics to characterize the set:


\todo{
geographic distribution
avg size 
avg contributions
avg contribs

georeferencing performance

qualification for each hypothesis
}


\section{H1 - H4 Article creation}
% H1 susceptible to outliers


\section{H5 - H8 Participation}

\section{H9/H10 Survival}

% H10 when locals low in the beginning but high text survival, its not first mover advantage, but better text quality.
