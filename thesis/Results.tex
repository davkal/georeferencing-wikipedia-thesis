%************************************************
\chapter{Results}\label{ch:results}
%************************************************

This chapter present the quantitative analysis of a set of articles of political events from the English Wikipedia.
First, I will describe the \nameref{sec:datasets} and the difficulties in picking the ``right'' category of articles.
Then, I will explain the results of the application's analysis of the data set and their relevance for the hypotheses.

\section{Data set}\label{sec:datasets}

Unfortunately, \term{political events} are a rather vague concept judging by the contents of the Wikipedia category of the same name\footurl{http://en.wikipedia.org/wiki/Category:Political_events}{2012}{02}{01}.
It includes only one page at the top-level, \term{No Berlusconi day}.
It contains numerous subcategories, however, ranging from \term{Political riots} and \term{Protests} to \term{Political party assemblies}.

In contrast to the categories system, templates offer a way to identify more homogenous groups of articles.\footnote{See also \nameref{sub:templates}.}
Articles that embed a template, e.g. \term{Infobox civil conflict}, can be identified and grouped, resulting in a set of articles that are very likely to treat civil conflicts. 

In order to find a suitable set of articles, I will combine both approaches.
From the English Wikipedia I will handpick categories and templates and combine their articles into one data set.

Since this process involves a manual selection, articles are not selected at random.
As a result, statistics derived from the analysis cannot be representative of all articles treating political events---let alone the complete corpus of the English Wikipedia.
On the bright side, the handpicked categories contain articles that were themselves manually chosen by the community to be included those categories.

\subsection{Selected categories and templates}

Wikipedia's category system suffers from several forms of overcategorization.\footnote{For an overview of issues see \weburl{http://en.wikipedia.org/wiki/Wikipedia:OCAT}{2012}{02}{04}} 
Some category sub-trees are too fine-grained and have numerous  levels with only a small number articles.
When moving along the tree in search for articles of political conflicts, the potential to go astray is rather high, e.g. the category \term{Protest}\footurl{http://en.wikipedia.org/wiki/Category:Protests}{2012}{02}{04} has the subcategory \term{Protest songs}.
There is also considerable overlap between sub-trees, e.g. \term{Protests by country} and \term{Protests by year}.
As a compromise, the category traversal will only yield the pages of a given category and all pages of its subcategories.

\term{Protests} and its subcategories are exhaustive and produce a handful of candidates.
To these I will add the category \term{Arab Spring} and the template \term{Infobox civil conflict}\footnote{This template is also the preferred infobox template of the Occupy movement. \weburl{http://en.wikipedia.org/wiki/Occupy_movement}{2012}{02}{04}} to assert that the articles detailed in \nameref{ch:experiment} are part of the data set.
The final data set \entity{conflictsraw} combines articles from the following categories and templates:

\begin{itemize}
  \item Protests by year % good, 45.5 % of 143
  \item Protests by country
  \item 2000s riots by year
  \item 2010s riots by year
  \item Arab Spring
  \item Infobox civil conflict (template)
\end{itemize}

Other notable categories that I investigated but not included, are 
\term{21st-century conflicts}\footnote{Only 10\% of the 833 articles would have qualified. \weburl{http://en.wikipedia.org/wiki/Category:21st-century_conflicts}{2012}{02}{04}}, 
\term{Revolutions categories}\footnote{This category and its subcategories contain too many articles about people. \weburl{http://en.wikipedia.org/wiki/Category:Revolutions_by_country}{2012}{02}{04}}, as well as the templates 
\term{Infobox military conflict}\footnote{A huge set of 10290 articles that is dominated by historic battles. Less than 6\% of these articles would have qualified. \weburl{http://en.wikipedia.org/wiki/Template:Infobox_military_conflict}{2012}{02}{04}} and 
\term{Infobox election}\footnote{These articles are usually created long before the election date, as elections tend to be scheduled. \weburl{http://en.wikipedia.org/wiki/Template:Infobox_election}{2012}{02}{04}}.


\subsection{Data set preparation}

When articles for \entity{conflictsraw} are retrieved, they are only added if they are not already in the set. 
Therefore, all articles in the set are unique.
The titles of all articles in \entity{conflictsraw} are listed in the appendix, see \nameref{sec:conflictsraw}.

Then, the all articles in \entity{conflictsraw} are tested whether they fulfill the \nameref{sub:articlerequirements} for an analysis.
Of the 742 articles in \entity{conflictsraw} 334 passed this first test.
The main reason for rejection was an event date before 2002. 
For the distribution of failed requirements \imgref{group-requirements}.
All articles that fulfill the requirements are included in the data set \entity{conflicts}, see \nameref{sec:conflicts} in the appendix for a list of titles.

\img{group-requirements}{Distribution of failed article requirements}


%\subsection{Political article vs. place article}
%
%\todo{
%To compare the date with spatial articles preferred in previous studies, a reference group is introduced
%template info box settlement is used by towns all over the world
%population is big 261,000+ articles, pick a random sample
%compare locating performance, sd metrics.
%}

\section{Characteristics of the set}

Each article in the set has a location attribute. 
Therefore the spatial distribution of these articles can be plotted on a map.

\imgwide{group-map}{Spatial distribution of articles in data set \entity{conflicts}.}

During the data set analysis I measured the similar metrics as in the \nameref{ch:experiment}.
Their means across all articles are shown in table \ref{groupmetrics}.

\ctable[
    caption = Metrics (mean) for articles in \entity{conflicts},
    	label   = groupmetrics,
pos=h,doinside=\small ]{Xlll}{
%    \tnote{Only applicable to a group of articles.}
%    \tnote[b]{Unable to locate creator.}
%    \tnote[c]{Share of located contributions.}
%    \tnote[d]{No correlation found.}
%    \tnote[e]{A negative date is the result of an article being merged with another article treating earlier events.}
 }{\toprule 
%    						& \tableheadline{Egypt-Rev} 
%    								& \tableheadline{Libya-war} 
%											& \tableheadline{Bahrain-up}\\ \midrule
    Age 				&  693.6d\\
    Size (\kb)			&  26.3\\
    Revisions 			&  303.5 \\
    Contributors\tmark		&  81.8\\
     -- anonymous		&  37.3\\
    \midrule
    Located contributors (\%) & 49.8 \\
    -- anonymous (\%)		& 40.0 \\
    -- by \abbr{PMCU}	(\%) &  5.2 \\
    -- by user page (\%)	& 4.6 \\
    \midrule
    \emph{Cumulative} \\
    Unique countries of origin	& 8.8 \\
    Signature distance (km) 	&  4,271.3\\
    \midrule
    \emph{Present in latest revision} \\
    Unique countries of origin 	& 5.4 \\
    Signature dist. e.surv (km) 	& 4,465.1\\
    Signature dist. t.surv (km) 	& 4374.6 \\
     e.surv index 				& 105.6 \\
     t.surv index		&  106.3\\
    Located text (\%) 	& 33.9 \\
    \bottomrule
}


For each of the hypotheses the articles in the set are tested whether they qualify, i.e. have enough data for an analysis, \imgref{group-qualifications}. 
The requirements for each hypothesis are defined in \nameref{hypothesesanalysis}.
%The test for \hyp{3} to the requirement is the most restrictive as it rules out events from countries with English as an official language.
The tests to qualify for \hyp{5} and \hyp{6} are the most restrictive as they require a certain number of revisions in the early interval of the event.
Also, the test for \hyp{3} to the requirement is quite restrictive as it rules out events from countries with English as an official language.
The qualification for \hyp{4} requires the article creator to be located, that seems to be also rarely the case.

\img{group-qualifications}{Article qualification for each hypothesis}

\section{H1 -- H4: Article creation}

On average, event articles are created 19.7 days after the start date of the event.
For articles that only have a month resolution, the average  is 223.1 days.
These average delays are longer than the stated limit of 7 days and 30 days, respectively.
However, the frequency distribution of delays shows a high number of articles publish with only a short delay, \imgref{group-h1}. 

\img{group-h1}{Distribution of creation delays in days}

There is only a slight correlation between recentness and creation delay ($R = 0.1$), suggesting the opposite of \hyp{1}.
 The delay after which an article is created is more likely to be bigger the more recent an article is.
For a scatter plot of  creation delay against creation date, \imgref{group-h2}.
The data for \hyp{1} and \hyp{2} has been cleared of outliers.\footnote{The lowest value still within 1.5 \ac{IQR} of the lower quartile, and the highest datum still within 1.5 \ac{IQR} of the upper quartile.}

\img{group-h2}{Creation delays over time}

Of the 62 articles qualified for \hyp{3}, 64\% (25) were created first in the English Wikipedia.  
The second most popular language in the set to first publish an article in, was Arabic, \imgref{group-h3}
For \hyp{3}, only articles qualified that treat events in countries that do not have English as an official language. 
The fact that the majority of these articles is published in English, suggests that article creators may prefer a global reach over sharing information with fellow citizen.

\img{group-h3}{Language editions chosen for first article}

\hyp{4}, namely, that articles about an event in a country are created by a resident, cannot be confirmed. 
The creator of an article could only be located for 79 articles (11\%) in the set. 
Among those, 89\% came from a different country, \imgref{group-h4}.

\img{group-h4}{Creator localness}

\section{H5 -- H8: Participation}

Of the 71 articles that qualified for \hyp{5}, the average share of anonymous contributions in the beginning is 20\%.
Therefore, \hyp{5} cannot be confirmed.
The highest share in the set was 50\%, see figure \ref{fig:group-h5}.

For \hyp{6}, even fewer articles qualified (n = 42).
The mean share of local contributions in the beginning is 10\%, thereby rejecting \hyp{6}.
See figure \ref{fig:group-h6} for the distribution.

\boximg{group-h5}{ Anon. }{group-h6}{Locals}{group-h56}{Distribution of contributor ratios in early revisions}

Over time, the divisions anonymous vs registered and local vs distant do not seem to matter much as the mean correlation coefficient is 0 for both cases, \imgref{group-h78}.
Therefore, \hyp{7} and \hyp{8} cannot be confirmed (n = 119).

\boximg{group-h7}{Anon.}{group-h8}{Local}{group-h78}{Distribution of correlation coefficients for the share of anonymous and local authors, respectively}


\section{H9 -- H10: Text survival}

Across the set, there is only a slight tendency for local content to be overrepresented. 
The mean \term{e.surv index} is 97; the mean \term{t.surv index} is 98.
These index values a only slightly smaller than the baseline of 100.
Therefore, \hyp{9} and \hyp{10} cannot be confirmed.
The data for \hyp{9} and \hyp{10} has been cleared of outliers using a very broad range.\footnote{The lowest value still within 4 \ac{IQR} of the lower quartile, and the highest datum still within 4 \ac{IQR} of the upper quartile.}
The box plot shows a number of outliers\footnote{These outliers are defined by the box plot. The lowest value is still within 1.5 \ac{IQR} of the lower quartile, and the highest datum still within 4 \ac{IQR} of the upper quartile.}, suggesting that there are articles that are dominated by local text, see figure \ref{fig:group-h910}.

\boximg{group-h9}{$\overline{I_{\text{e.surv}}}$}{group-h10}{$\overline{I_{\text{t.surv}}}$}{group-h910}{Distribution of means of localness indexes}


% H10 when locals low in the beginning but high text survival, its not first mover advantage, but better text quality.
