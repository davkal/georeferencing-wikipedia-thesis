%********************************************************************
% Errata
%*******************************************************
\chapter{Errata}

The following mistakes need to be corrected from the version that I submitted as my diploma thesis:

\section{Minor}

In \nameref{ch:apparatus} (ch. \ref{ch:apparatus}), the \nth{4} paragraph should read: ``The application can be accessed over the internet.''

In \nameref{sub:proximitymetrics} (subsec. \ref{sub:proximitymetrics}), in the last paragraph, the mean share of the located text is actually 26.2\%, instead of 23.8\%.

In \nameref{sec:characteristics} (sec. \ref{sec:characteristics}), in table \ref{groupmetrics}, the footnote $^{a}$ is missing and should read ``Excluding bots.''.

\todo{s/mean/median/ in 6.4 and 6.5}

\section{Omissions}

In \nameref{sub:sigdistimpl} (subsec. \ref{sub:sigdistimpl}), listing \ref{signaturedistanceall} presents an algorithm to efficiently calculate the signature distance for all revisions of an article.
It uses an optimization called memoization for which a recursive formula is being used. 
The step of converting the iterative declaration of the signature distance $D(\alpha)$ into a recursive declaration was omitted and is added here for completion.

Using the same notation, let $\alpha$ be an article, $\rho$ be an author, and $\pi$ be a revision.
For a sample of articles $S$ let $A$ be the set all articles in the sample, $P$ be the set of all located authors in the sample, and $\Pi$ the set of all revisions in the sample.
\begin{align*}
A &= \{\alpha : \alpha \in S\} &P &= \{\rho : \rho \in S\} & \Pi &= \{\pi : \pi \in S\}
\end{align*}
Further, let $\eta(\rho,\alpha)$ denote the contribution(s) of author $\rho$ to article $\alpha$.
Then, $N(\alpha)$ are all contribution(s) to article $\alpha$ while $P(\alpha)$ are the author(s) who have made contributions to article $\alpha$:
\begin{align*}
N(\alpha) &= \{\eta(\rho,\alpha) : \rho \in P\} & P(\alpha) &= \{\rho : \rho \in N(\alpha)\}
\end{align*}

In addition, let $\gamma(\pi,\alpha)$ denote whether $\pi$ is a revision of article $\alpha$.
Then, $\Gamma(\alpha)$ are all revisions to article $\alpha$:

\begin{align*}
\Gamma(\alpha) &= \{\gamma(\pi,\alpha) : \pi \in \Pi\}
\end{align*}

Since a contribution creates a revision, the total number $n$ of contributions equals the number of revisions to article $\alpha$:

\begin{align*}
 n = |\Gamma(\alpha)| = |N(\alpha)|
\end{align*}

Let further $\delta(\rho,\alpha)$ be the geodesic distance between author $\rho$ and article $\alpha$.
Additionally, let $\zeta(\pi)$ denote  the author $\rho$ of revision $\pi$ to article $\alpha$.
It follows that the distance of a revision to the article is the same as the distance between the article and that revision's author:

\begin{align*}
\rho &= \zeta(\pi) \, \RA \,   \delta(\zeta(\pi),\alpha) = \delta(\rho,\alpha) 
\end{align*}

In the original signature distance $D(\alpha)$, all contributions by an author were summed up by $\eta(\rho,\alpha)$.
Considering each of these contributions as a single revision $\pi$, the signature distance can also be  expressed as follows:
\begin{align*}
D(\alpha) &= \sum_{\forall \rho \in P(\alpha)} \frac{|\eta(\rho,\alpha)| \cdot \delta(\rho,\alpha)}{|N(\alpha)|} \\
 &=  \sum_{\forall \pi \in \Gamma(\alpha)} \frac{\delta(\zeta(\pi),\alpha)}{|\Gamma(\alpha)|}
 \end{align*}

When expanded, it becomes obvious that the signature distance for $n = |\Gamma(\alpha)|$ revisions contains the signature distance for the previous $n-1$ revisions:
 
\begin{align*}
D(\alpha) &=  \frac{\delta(\zeta(\pi_{1}),\alpha)  + \ldots + \delta(\zeta(\pi_{n-1}),\alpha) + \delta(\zeta(\pi_{n}),\alpha)}{n} \\
 &=  \frac{\delta(\zeta(\pi_{1}),\alpha)  + \ldots + \delta(\zeta(\pi_{n-1}),\alpha)}{n} + \frac{\delta(\zeta(\pi_{n}),\alpha)}{n} \\
 &=  \frac{(\delta(\zeta(\pi_{1}),\alpha)  + \ldots + \delta(\zeta(\pi_{n-1}),\alpha))(n-1)}{n(n-1)} + \frac{\delta(\zeta(\pi_{n}),\alpha)}{n} \\
 &=  \frac{n-1}{n} \cdot \frac{\delta(\zeta(\pi_{1}),\alpha)  + \ldots + \delta(\zeta(\pi_{n-1}),\alpha)}{n-1} + \frac{\delta(\zeta(\pi_{n}),\alpha)}{n} 
\end{align*}

Hence, the signature distance can be restated recursively:

\begin{align}
D(\alpha, \pi_{1}) &= \delta(\zeta(\pi_{1}),\alpha)\label{eqn:sigdistrecursive1}\\
D(\alpha, \pi_{n}) &=  D(\alpha, \pi_{n-1}) \cdot \frac{n-1}{n} + \frac{\delta(\zeta(\pi_{n}),\alpha)}{n} \label{eqn:sigdistrecursive2}
\end{align}

The distance for the first revision, \ref{eqn:sigdistrecursive1}, is the base case where the signature distance of the article is only the distance between the author $\rho$ of revision $\pi_{1}$ and the article $\alpha$.
For subsequent revisions, the signature distance can be calculated by using the result for the previous revisions, \ref{eqn:sigdistrecursive2}.


\section{Quantitative analysis}

A postmortem of the quantitative analysis data, used in \nameref{ch:results} (ch. \ref{ch:results}), revealed that around 15\% of articles were illegitimately rejected\footnote{These articles passed most of the initial \nameref{sub:articlerequirements} (subsec. \ref{sub:articlerequirements}). But then, due to a Toolserver unavailability, the authorship could not be determined and the article was rejected.}

A re-run of the analysis was done on \dformat{2012}{02}{07}\footnote{The final run for the thesis was done on \dformat{2012}{02}{05}.}.
There, a net\footnote{Two old articles, \term{Homs Bombardment (2012)} and \term{Homs massacre} were merged into \term{February 2012 bombardment of Homs}.} five articles were added to the categories and templates that made up the set \entity{conflictsraw}:

\begin{multicols}{2}\scriptsize
\setlength{\parindent}{0pt}
1981 General strike in Bielsko-Biała\\
2011-2012 Syrian civil war\\
Castellammarese War\\
February 2012 bombardment of Homs\\
Great Nordic Biker War\\
Masrena
\end{multicols}

The new raw set \entity{confictsraw-b} consists of 746 articles. 
I denote these new articles together with the previously rejected articles as \entity{skipped}.
The resulting set for analysis \entity{conflicts-b} thus contains the articles of \entity{conflicts} plus the articles of \entity{skipped}.
The  articles in \entity{skipped} are:

\begin{multicols}{3}\tiny
\setlength{\parindent}{0pt}
2006 Dalit protests in Maharashtra\\
2006 Indian anti-reservation protests\\
2006 Islamist demonstration outside the Embassy of Denmark in London\\
2006 Oaxaca protests\\
2006 protests in Hungary\\
2006 student protests in Chile\\
2006 youth protests in France\\
2007 Bersih rally\\
2007 Burmese anti-government protests\\
2007 HINDRAF rally\\
2007 MacArthur Park rallies\\
2007 Macau labour protest\\
2007 Macau transfer of sovereignty anniversary protest\\
2007 Russian protests\\
2008 Armenian presidential election protests\\
2008 Cameroonian anti-government protests\\
2008 Icelandic lorry driver protests\\
2008 Otago NORML protests\\
2008–2010 Thai political crisis\\
2009 Ashura protests\\
2009 G-20 London summit protests\\
2009 Georgian demonstrations\\
2009 Guinea protest\\
2009 Luquan protest\\
2009 May Day protests\\
2009 Tamil diaspora protests in Canada\\
2009 student protests in Austria\\
2009 student protests in Croatia\\
2009–2010 Iranian election protests\\
2010 Air Untuk Rakyat rally\\
2010 Canada anti-prorogation protests\\
2010 Catalan autonomy protest\\
2010 G-20 Toronto summit protests\\
2010 Macau labour protest\\
2010 Macau transfer of sovereignty anniversary protest\\
2010 Thai political protests\\
2010 Xinfa aluminum plant protest\\
2010–2011 Greek protests\\
2010–2011 Senegalese protests\\
2010–2012 Algerian protests\\
2011 Albanian opposition demonstrations\\
2011 Armenian protests\\
2011 Chilean protests\\
2011 Chinese pro-democracy protests\\
2011 Djiboutian protests\\
2011 Indian anti-corruption movement\\
2011 Iranian protests\\
2011 Iraqi protests\\
2011 Israeli social justice protests\\
2011 Kurdish protests in Turkey\\
2011 Lebanese protests\\
2011 London anti-cuts protest\\
2011 Malawian protests\\
2011 Maldivian protests\\
2011 Mexican protests\\
2011 Nanchang mass suicide protest\\
2011 Omani protests\\
2011 Rome demonstration\\
2011 Spanish protests\\
2011 Sudanese protests\\
2011 United Kingdom anti-austerity protests\\
2011 United States public employee protests\\
2011 Western Saharan protests\\
2011 Wisconsin protests\\
2011 Yunnan protest\\
2011–2012 Jordanian protests\\
2011–2012 Kuwaiti protests\\
2011–2012 Moroccan protests\\
2011–2012 Saudi Arabian protests\\
2011–2012 Yemeni uprising\\
2012 Romanian protests\\
517 Protest\\
Aboriginal Day of Action\\
Anti-Roma protests in Bulgaria 2011\\
Arab Spring\\
Berkeley Marine Corps Recruiting Center protests\\
Cottage cheese boycott\\
Cow head protests\\
Crimean anti-NATO protests of 2006\\
Dazhou protests of 2007\\
December 2005 protest for democracy in Hong Kong\\
Dongzhou protests\\
February 2012 bombardment of Homs\\
Glasgow school closures protest, 2009\\
January 27, 2007 anti-war protest\\
March 19, 2008 anti-war protest\\
May 2007 RCTV protests\\
Mid 2011 Telangana protests\\
November 15, 2008 anti-Proposition 8 protests\\
Occupy Charlottesville\\
Occupy Nigeria\\
Occupy movement\\
Port of Tacoma protests, March 2007\\
Protests following the 2011 Russian elections\\
Protests regarding 2008 South Ossetia war\\
Republic Protests\\
September 15, 2007 anti-war protest\\
Sharad Pawar slapping incident\\
SlutWalk\\
Taxpayer March on Washington\\
Tea Party protests\\
Tripoli protests and clashes (February 2011)\\
Zakariya Rashid Hassan al-Ashiri\\
Zhejiang solar panel plant protest
\end{multicols}

The new set \entity{conflicts-b} contains 380 articles.
Although bigger in size, the new set's effect on the outcome of the quantitative analysis is negligible.
This is due to the fact that articles are shuffled within the set after retrieval.\footnote{As a result, the run on \entity{conflicts} was done, in effect,  on a random sample.}
The same metrics as in table \ref{groupmetrics} have been measured for \entity{conflicts-b}.
For a comparison, the mean values for both sets are shown in table \ref{groupmetricsE}.

\ctable[
    caption = Metrics (mean) for articles in \entity{conflicts} and \entity{conflicts-b},
    	label   = groupmetricsE,
pos=h,doinside=\small ]{Xrr}{
    \tnote{Excluding bots.}
 }{\toprule 
 					& \tableheadline{conflicts} 
									& \tableheadline{conflicts-b}\\ \midrule
    Age 						&  693.6d 		& 728.2d\\
    Size (\kb)					&  26.3 		& 28.9\\
    Revisions 					&  303.5 		& 357.7 \\
    Contributors\tmark				&  81.8		& 95.9 \\
     -- anonymous				&  37.3		& 44.2 \\
    \midrule
    Located contributors (\%) 		& 49.8 		& 48.7 \\
    -- anonymous (\%)				& 40.0 		& 41.5 \\
    -- by \abbr{PMCU}	(\%) 			&  5.2 		& 4.2 \\
    -- by user page (\%)			& 4.6 		& 3.0 \\
    \midrule
    \emph{Cumulative} \\
    Unique countries of origin		& 8.8 		& 9.6 \\
    Signature distance (km) 		& 4,271.3		& 4,252.7\\
    \midrule
    \emph{Present in latest revision} \\
    Unique countries of origin 		& 5.4 		& 5.9 \\
    Signature dist. e.surv (km) 		& 4,465.1		& 4,330.4\\
    Signature dist. t.surv (km) 		& 4374.6 		& 4,256.4\\
     e.surv index 					& 105.6 		& 101.7\\
     t.surv index					&  106.3		& 103.8 \\
    Located text (\%) 				& 33.9 		& 33.3 \\
    \bottomrule
}

At first glance, the articles in \entity{skipped} seem to contribute especially old articles to \entity{conflicts-b}.
These older articles tend to have a higher number of revisions as well as contributors, raising the mean values for age, revisions, and contributors.
The differences for the proximity metrics are minor.